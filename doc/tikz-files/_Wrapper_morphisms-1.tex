
\nonstopmode
\documentclass{standalone}

% This document contains the TikZ-header for all our LaTeX-computations.
% It especially contains all global graphic parameters.

\usepackage{amsmath, amssymb, amsfonts} % Standard Math-stuff

\usepackage{tikz}
\usetikzlibrary{calc}
\usetikzlibrary{positioning}

% Define a text=none option for nodes that ignores the given text, from
% https://tex.stackexchange.com/questions/59354/no-text-none-in-tikz
\makeatletter
\newif\iftikz@node@phantom
\tikzset{
  phantom/.is if=tikz@node@phantom,
  text/.code=%
    \edef\tikz@temp{#1}%
    \ifx\tikz@temp\tikz@nonetext
      \tikz@node@phantomtrue
    \else
      \tikz@node@phantomfalse
      \let\tikz@textcolor\tikz@temp
    \fi
}
\usepackage{etoolbox}
\patchcmd\tikz@fig@continue{\tikz@node@transformations}{%
  \iftikz@node@phantom
    \setbox\pgfnodeparttextbox\hbox{}
  \fi\tikz@node@transformations}{}{}
\makeatother

% Now we define the global styles
% The global styles are defined nestedly. You have to give your tikzpicture
% the global options [vertexStyle, edgeStyle, faceStyle] to activate them.
% 
% You can disable labels by using the option nolabels, i.e. 
% vertexStyle=nolabels to deactivate vertex labels.
%
% If you want to have a specific style for your picture, you can also use
% this specific meta-style instead of the general style. For example if you
% want to use double edges in one single picture - no matter the style of
% the rest of the document - you can use edgeDouble instead of edgeStyle.
%
% To set the default style, modify the vertexStyle/.default entry.

% Vertex styles
\tikzset{ 
    vertexNodePlain/.style = {fill=gray, shape=circle, inner sep=0pt, minimum size=2pt, text=none},
    vertexPlain/labels/.style = {
        vertexNode/.style={vertexNodePlain},
        vertexLabel/.style={gray}
    },
    vertexPlain/nolabels/.style = {
        vertexNode/.style={vertexNodePlain},
        vertexLabel/.style={text=none}
    },
    vertexPlain/.style = vertexPlain/#1,
    vertexPlain/.default=labels
}
\tikzset{
    vertexNodeNormal/.style = {fill=blue, shape=circle, inner sep=0pt, minimum size=4pt, text=none},
    vertexNormal/labels/.style = {
        vertexNode/.style={vertexNodeNormal},
        vertexLabel/.style={blue}
    },
    vertexNormal/nolabels/.style = {
        vertexNode/.style={vertexNodeNormal},
        vertexLabel/.style={text=none}
    },
    vertexNormal/.style = vertexNormal/#1,
    vertexNormal/.default=labels
}
\tikzset{
    vertexNodeBall/.style = {shape=circle, ball color=orange, inner sep=2pt, outer sep=0pt, minimum size=3pt},
    vertexBall/labels/.style = {
        vertexNode/.style={vertexNodeBall, text=black},
        vertexLabel/.style={text=none}
    },
    vertexBall/nolabels/.style = {
        vertexNode/.style={vertexNodeBall, text=none},
        vertexLabel/.style={text=none}
    },
    vertexBall/.style = vertexBall/#1,
    vertexBall/.default=labels
}
\tikzset{ 
    vertexStyle/.style={vertexNormal=#1},
    vertexStyle/.default = labels
}


% 1) position of the vertex
% 2) relative position of the node
% 3) name of the vertex
\newcommand{\vertexLabelR}[3]{
    \node[vertexLabel, #2] at (#1) {#3};
}
% 1) position of the vertex
% 2) absolute position of the node
% 3) name of the vertex
\newcommand{\vertexLabelA}[3]{
    \node[vertexLabel] at (#2) {#3};
}


% Edge styles
% If you have trouble with the double-lines overlapping, this might (?) help:
% https://tex.stackexchange.com/questions/288159/closing-the-ends-of-double-line-in-tikz
\tikzset{
    edgeLinePlain/.style={line join=round},
    edgePlain/labels/.style = {
        edge/.style={edgeLinePlain},
        edgeLabel/.style={fill=blue!20!white}
    },
    edgePlain/nolabels/.style = {
        edge/.style={edgeLinePlain},
        edgeLabel/.style={text=none}
    },
    edgePlain/.style = edgePlain/#1,
    edgePlain/.default = labels
}
\tikzset{
    edgeLineDouble/.style = {thin, double=gray!90!white, double distance=.3pt, line join=round},
    edgeDouble/labels/.style = {
        edge/.style = {edgeLineDouble},
        edgeLabel/.style = {fill=blue!20!white}
    },
    edgeDouble/nolabels/.style = {
        edge/.style = {edgeLineDouble},
        edgeLabel/.style = {text=none}
    },
    edgeDouble/.style = edgeDouble/#1,
    edgeDouble/.default = labels
}
\tikzset{
    edgeStyle/.style = {edgePlain=#1},
    edgeStyle/.default = labels
}

% Face styles
% Here we have an exception - the style face is always defined.
% 
\newcommand{\faceColorY}{yellow!60!white}   % yellow
\newcommand{\faceColorB}{blue!60!white}     % blue
\newcommand{\faceColorP}{cyan!60}           % purple
\newcommand{\faceColorR}{red!60!white}      % red
\newcommand{\faceColorG}{green!60!white}    % green
\newcommand{\faceColorO}{orange!50!yellow!70!white} % orange

\newcommand{\faceColor}{\faceColorY}
\newcommand{\faceColorSwap}{\faceColorB}
\tikzset{
    face/.style = {fill=#1},
    face/.default = \faceColor,
    faceY/.style = {face=\faceColorY},
    faceB/.style = {face=\faceColorB},
    faceP/.style = {face=\faceColorP},
    faceR/.style = {face=\faceColorR},
    faceG/.style = {face=\faceColorG},
    faceO/.style = {face=\faceColorO}
}
\tikzset{
    faceStyle/labels/.style = {
        faceNode/.style = {}
    },
    faceStyle/nolabels/.style = {
        faceNode/.style = {text=none}
    },
    faceStyle/.style = faceStyle/#1,
    faceStyle/.default = labels
}
\tikzset{ face/.style={fill=#1} }
\tikzset{ faceSwap/.code=
    \ifdefined\swapColours
        {face=\faceColorSwap}
    \else
        {face=\faceColor}
    \fi
}

% Sometimes we want to implement different behaviour for the generated 
% HTML-pictures (for example, shading is not supported in HTML).
% For that we define a macro to check whether we run the code with
% htlatex. The code comes from 
% https://tex.stackexchange.com/questions/93852/what-is-the-correct-way-to-check-for-latex-pdflatex-and-html-in-the-same-latex
\makeatletter
\edef\texforht{TT\noexpand\fi
  \@ifpackageloaded{tex4ht}
    {\noexpand\iftrue}
    {\noexpand\iffalse}}
\makeatother


\usepackage{hyperref}


\def\pgfsysdriver{pgfsys-tex4ht.def}

\begin{document}
       \begin{tikzpicture}[vertexStyle, edgeStyle, faceStyle]
         
\newcommand{\setupCoord}{
    \coordinate (Z) at (0,0);
    \foreach \i in {1,...,6}{
        \coordinate (A\i) at (180-60*\i:2.5);
    }
}


\begin{scope}[shift={(-3.5,0)}]
    \setupCoord

    \draw[edge,face]
        (Z) -- (A1) -- node[edgeLabel]{1} (A2) -- cycle
        (Z) -- node[edgeLabel]{9} (A2) -- node[edgeLabel]{2} (A3) -- cycle
        (Z) -- node[edgeLabel]{10} (A3) -- node[edgeLabel]{3} (A4) -- cycle
        (Z) -- node[edgeLabel]{11} (A4) -- node[edgeLabel]{4} (A5) -- cycle
        (Z) -- node[edgeLabel]{12} (A5) -- node[edgeLabel]{5} (A6) -- cycle
        (Z) -- node[edgeLabel]{13} (A6) -- node[edgeLabel]{6} (A1) -- node[edgeLabel]{8} (Z);

    \foreach \p/\r/\n in {1/above left/1, 2/above right/2, 3/right/3, 4/below right/4, 5/below left/5, 6/left/6}{
        \vertexLabelR{A\p}{\r}{\n}
    }
    \vertexLabelR{Z}{above}{8}

    \foreach \i/\j/\n in {1/2/I, 2/3/II, 3/4/III, 4/5/IV, 5/6/V, 6/1/VI}{
        \node[faceLabel] at (barycentric cs:A\i=1,A\j=1,Z=1) {$\n$};
    }
\end{scope}


\node at (0,0) {$\rightarrow$};


\begin{scope}[shift={(2.5,0)}]
    \setupCoord

    \draw[edge,face]
        (Z) -- node[edgeLabel]{5} (A1) -- node[edgeLabel]{1} (A2) -- cycle
        (Z) -- node[edgeLabel]{6} (A2) -- node[edgeLabel]{2} (A3) -- cycle
        (Z) -- node[edgeLabel]{7} (A3) -- node[edgeLabel]{3} (A4) -- node[edgeLabel]{5} (Z);

    \foreach \p/\r/\n in {1/above left/1, 2/above right/2, 3/right/3, 4/below right/1}{
        \vertexLabelR{A\p}{\r}{\n}
    }
    \vertexLabelR{Z}{below left}{5}

    \foreach \i/\j/\n in {1/2/I, 2/3/II, 3/4/III}{
        \node[faceLabel] at (barycentric cs:A\i=1,A\j=1,Z=1) {$\n$};
    }
\end{scope}

       \end{tikzpicture}
     </Alt>
 @ExampleSession
 gap> six := SimplicialSurfaceByDownwardIncidence(
 >     [[1,2],[2,3],[3,4],[4,5],[5,6],[6,1],,[1,8],[2,8],[3,8],[4,8],[5,8],[6,8]],
 >     [[1,8,9],[2,9,10],[3,10,11],[4,11,12],[5,12,13],[6,13,8]]);;
 gap> three := SimplicialSurfaceByDownwardIncidence(
 >     [[1,2],[2,3],[3,1],,[1,5],[2,5],[3,5]], [[1,5,6],[2,6,7],[3,7,5]]);;
 gap> mor_6_to_6 := PolygonalIdentityMorphism(six);;
 gap> mor_3_to_3 := PolygonalMorphismByLists(three, three,
 >       [2,3,1,,5], [2,3,1,,6,7,5], [2,3,1]);;
 gap> mor_6_to_3 := PolygonalMorphismByLists(six, three, 
 >    [1,2,3,1,2,3,,5], [1,2,3,1,2,3,,5,6,7,5,6,7], [1,2,3,1,2,3]);;
 gap> comp := CompositionMapping2(mor_3_to_3, mor_6_to_3);;
 gap> VertexMapAsImageList(comp);
 [ 2, 3, 1, 2, 3, 1,, 5 ]
 gap> EdgeMapAsImageList(comp);
 [ 2, 3, 1, 2, 3, 1,, 6, 7, 5, 6, 7, 5 ]
 gap> FaceMapAsImageList(comp);
 [ 2, 3, 1, 2, 3, 1 ]
 gap> CompositionMapping2(mor_6_to_3, mor_6_to_6) = mor_6_to_3;
 true
 gap> CompositionMapping(mor_3_to_3, mor_6_to_3, mor_6_to_6) = comp;
 true
 @EndExampleSession

   </Description>
 </ManSection>



 <ManSection Label="InversePolygonalMorphism">
   <Attr Name="InversePolygonalMorphism" Arg="isoMor" 
      Label="for IsPolygonalMorphism and IsBijective"
      Comm="Construct the inverse polygonal morphism from a bijective polygonal morphism"/>
   <Oper Name="Inverse" Arg="autoMor" 
      Label="for a bijective polygonal morphisms with identical source and range"
      Comm="Construct the inverse polygonal morphism from a bijective polygonal morphism with identical source and range"/>
   <Attr Name="InverseGeneralMapping" Arg="isoMor" 
      Label="for a bijective polygonal morphism"
      Comm="Construct the inverse polygonal morphism from a bijective polygonal morphism"/>
   <Returns>A polygonal morphism</Returns>
   <Description>
     Given a bijective polygonal morphism, one can define its inverse,
     i.e. a polygonal morphism, in which <K>SourceComplex</K>
     (<Ref Subsect="SourceComplex"/>) and <K>RangeComplex</K>
     (<Ref Subsect="RangeComplex"/>) are switched.
     
     Due to the way in which GAP handles inverses and mappings (compare the
     introduction of section 
     <Ref Sect="Arithmetic Operations for General Mapping" BookName="Reference"/>), 
     the different 
     methods perform subtly different tasks:
     * <K>InversePolygonalMorphism</K>(<A>isoMor</A>) 
       constructs the expected inverse
       map, from <K>RangeComplex</K>(<A>isoMor</A>) to
       <K>SourceComplex</K>(<A>isoMor</A>).
     * <K>Inverse</K>(<A>autoMor</A>) <E>only</E> constructs this inverse, if 
       <K>SourceComplex</K>(<A>autoMor</A>) and
       <K>RangeComplex</K>(<A>autoMor</A>) coincide.
     * <K>InverseGeneralMapping</K> does the same
       as <K>InversePolygonalMorphism</K>, but might be subject to future
       change, if inverses are defined for non-bijective morphisms in
       the future.

     All of the methods throw errors if their requirements are not met.

     To illustrate, we use a relabelling of a five-umbrella.
      <Alt Only="HTML">
 &lt;br>&lt;img src="./images/_Wrapper_Image_PolygonalMorphism_FiveUmbrella-1.svg"> &lt;/img> &lt;br>
end{document}