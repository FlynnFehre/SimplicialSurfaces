
\nonstopmode
\documentclass{standalone}

% This document contains the TikZ-header for all our LaTeX-computations.
% It especially contains all global graphic parameters.

\usepackage{amsmath, amssymb, amsfonts} % Standard Math-stuff

\usepackage{tikz}
\usetikzlibrary{calc}
\usetikzlibrary{positioning}

% Define a text=none option for nodes that ignores the given text, from
% https://tex.stackexchange.com/questions/59354/no-text-none-in-tikz
\makeatletter
\newif\iftikz@node@phantom
\tikzset{
  phantom/.is if=tikz@node@phantom,
  text/.code=%
    \edef\tikz@temp{#1}%
    \ifx\tikz@temp\tikz@nonetext
      \tikz@node@phantomtrue
    \else
      \tikz@node@phantomfalse
      \let\tikz@textcolor\tikz@temp
    \fi
}
\usepackage{etoolbox}
\patchcmd\tikz@fig@continue{\tikz@node@transformations}{%
  \iftikz@node@phantom
    \setbox\pgfnodeparttextbox\hbox{}
  \fi\tikz@node@transformations}{}{}
\makeatother

% Now we define the global styles
% The global styles are defined nestedly. You have to give your tikzpicture
% the global options [vertexStyle, edgeStyle, faceStyle] to activate them.
% 
% You can disable labels by using the option nolabels, i.e. 
% vertexStyle=nolabels to deactivate vertex labels.
%
% If you want to have a specific style for your picture, you can also use
% this specific meta-style instead of the general style. For example if you
% want to use double edges in one single picture - no matter the style of
% the rest of the document - you can use edgeDouble instead of edgeStyle.
%
% To set the default style, modify the vertexStyle/.default entry.

% Vertex styles
\tikzset{ 
    vertexNodePlain/.style = {fill=gray, shape=circle, inner sep=0pt, minimum size=2pt, text=none},
    vertexPlain/labels/.style = {
        vertexNode/.style={vertexNodePlain},
        vertexLabel/.style={gray}
    },
    vertexPlain/nolabels/.style = {
        vertexNode/.style={vertexNodePlain},
        vertexLabel/.style={text=none}
    },
    vertexPlain/.style = vertexPlain/#1,
    vertexPlain/.default=labels
}
\tikzset{
    vertexNodeNormal/.style = {fill=blue, shape=circle, inner sep=0pt, minimum size=4pt, text=none},
    vertexNormal/labels/.style = {
        vertexNode/.style={vertexNodeNormal},
        vertexLabel/.style={blue}
    },
    vertexNormal/nolabels/.style = {
        vertexNode/.style={vertexNodeNormal},
        vertexLabel/.style={text=none}
    },
    vertexNormal/.style = vertexNormal/#1,
    vertexNormal/.default=labels
}
\tikzset{
    vertexNodeBall/.style = {shape=circle, ball color=orange, inner sep=2pt, outer sep=0pt, minimum size=3pt},
    vertexBall/labels/.style = {
        vertexNode/.style={vertexNodeBall, text=black},
        vertexLabel/.style={text=none}
    },
    vertexBall/nolabels/.style = {
        vertexNode/.style={vertexNodeBall, text=none},
        vertexLabel/.style={text=none}
    },
    vertexBall/.style = vertexBall/#1,
    vertexBall/.default=labels
}
\tikzset{ 
    vertexStyle/.style={vertexNormal=#1},
    vertexStyle/.default = labels
}


% 1) position of the vertex
% 2) relative position of the node
% 3) name of the vertex
\newcommand{\vertexLabelR}[3]{
    \node[vertexLabel, #2] at (#1) {#3};
}
% 1) position of the vertex
% 2) absolute position of the node
% 3) name of the vertex
\newcommand{\vertexLabelA}[3]{
    \node[vertexLabel] at (#2) {#3};
}


% Edge styles
% If you have trouble with the double-lines overlapping, this might (?) help:
% https://tex.stackexchange.com/questions/288159/closing-the-ends-of-double-line-in-tikz
\tikzset{
    edgeLinePlain/.style={line join=round},
    edgePlain/labels/.style = {
        edge/.style={edgeLinePlain},
        edgeLabel/.style={fill=blue!20!white}
    },
    edgePlain/nolabels/.style = {
        edge/.style={edgeLinePlain},
        edgeLabel/.style={text=none}
    },
    edgePlain/.style = edgePlain/#1,
    edgePlain/.default = labels
}
\tikzset{
    edgeLineDouble/.style = {thin, double=gray!90!white, double distance=.3pt, line join=round},
    edgeDouble/labels/.style = {
        edge/.style = {edgeLineDouble},
        edgeLabel/.style = {fill=blue!20!white}
    },
    edgeDouble/nolabels/.style = {
        edge/.style = {edgeLineDouble},
        edgeLabel/.style = {text=none}
    },
    edgeDouble/.style = edgeDouble/#1,
    edgeDouble/.default = labels
}
\tikzset{
    edgeStyle/.style = {edgePlain=#1},
    edgeStyle/.default = labels
}

% Face styles
% Here we have an exception - the style face is always defined.
% 
\newcommand{\faceColorY}{yellow!60!white}   % yellow
\newcommand{\faceColorB}{blue!60!white}     % blue
\newcommand{\faceColorP}{cyan!60}           % purple
\newcommand{\faceColorR}{red!60!white}      % red
\newcommand{\faceColorG}{green!60!white}    % green
\newcommand{\faceColorO}{orange!50!yellow!70!white} % orange

\newcommand{\faceColor}{\faceColorY}
\newcommand{\faceColorSwap}{\faceColorB}
\tikzset{
    face/.style = {fill=#1},
    face/.default = \faceColor,
    faceY/.style = {face=\faceColorY},
    faceB/.style = {face=\faceColorB},
    faceP/.style = {face=\faceColorP},
    faceR/.style = {face=\faceColorR},
    faceG/.style = {face=\faceColorG},
    faceO/.style = {face=\faceColorO}
}
\tikzset{
    faceStyle/labels/.style = {
        faceNode/.style = {}
    },
    faceStyle/nolabels/.style = {
        faceNode/.style = {text=none}
    },
    faceStyle/.style = faceStyle/#1,
    faceStyle/.default = labels
}
\tikzset{ face/.style={fill=#1} }
\tikzset{ faceSwap/.code=
    \ifdefined\swapColours
        {face=\faceColorSwap}
    \else
        {face=\faceColor}
    \fi
}

% Sometimes we want to implement different behaviour for the generated 
% HTML-pictures (for example, shading is not supported in HTML).
% For that we define a macro to check whether we run the code with
% htlatex. The code comes from 
% https://tex.stackexchange.com/questions/93852/what-is-the-correct-way-to-check-for-latex-pdflatex-and-html-in-the-same-latex
\makeatletter
\edef\texforht{TT\noexpand\fi
  \@ifpackageloaded{tex4ht}
    {\noexpand\iftrue}
    {\noexpand\iffalse}}
\makeatother


\usepackage{hyperref}


\def\pgfsysdriver{pgfsys-tex4ht.def}

\begin{document}
   	\begin{tikzpicture}[vertexStyle, edgeStyle=nolabels, faceStyle]
      	\def\r{2.5}

\coordinate (Z) at (0,0);
\foreach \i in {1,...,5}{
    \coordinate (P\i) at (90+\i*72:\r);
}

\draw[edge, face]
    (Z) -- (P4) -- node[edgeLabel]{6} (P5) -- cycle
    (Z) -- node[edgeLabel]{2} (P4) -- node[edgeLabel]{8} (P3) -- cycle
    (Z) -- node[edgeLabel]{3} (P3) -- node[edgeLabel]{9} (P2) -- cycle
    (Z) -- node[edgeLabel]{4} (P2) -- node[edgeLabel]{10} (P1) -- cycle
    (Z) -- node[edgeLabel]{5} (P1) -- node[edgeLabel]{7} (P5) -- node[edgeLabel]{1} cycle;

\foreach \p/\r/\n in {P1/left/11, P2/left/7, P3/right/5, P4/right/3, P5/above/2, Z/below/1}{
    \vertexLabelR{\p}{\r}{\n}
}

\foreach \p/\q/\n in {P1/P2/IV, P2/P3/III, P3/P4/II, P4/P5/I, P5/P1/V}{
    \node[faceLabel] at (barycentric cs:Z=1,\p=1,\q=1) {\n};
}

   	\end{tikzpicture}
 	</Alt>
 	@BeginExampleSession
 gap> countEdg := CounterOfEdges(fiveStar);
 counter of edges ([ [ 2, 2 ], [ 2, 5 ] ] degrees, and [ 5, 5 ] multiplicities)
 gap> ListCounter(countEdg);
 [ [ [ 2, 2 ], 5 ], [ [ 2, 5 ], 5 ] ]
 	@EndExampleSession
   </Description>
 </ManSection>

 <ManSection Label="CounterOfFaces">
   <Oper Name="CounterOfFaces" Arg="complex"
      Label="for IsTwistedPolygonalComplex"
      Comm="Construct a face counter from a twisted polygonal complex"/>
   <Returns>A Counter-&GAP;-object</Returns>
   <Prop Name="IsCounterOfFaces" Arg="object" Label="for IsObject"
      Comm="Check whether a given object is a face counter"/>
   <Returns><K>true</K> or <K>false</K></Returns>
   <Description>
      The method <K>CounterOfFaces</K> constructs a new face counter from
      a polygonal complex. The method <K>IsCounterOfFaces</K>
      checks if a given &GAP;-object represents such a face counter.
      The face counter saves the information about how many faces have vertices with the same degrees.
 	To get this information there are different possibilities.
 	For example, the method <A>ListCounter</A> (<Ref Subsect="ListCounter"/>) that
 	returns this information as a list of pairs <E>[degreeList, multiplicity]</E>, where
 	<E>multiplicity</E> counts the number of faces whose vertex degrees
 	match <E>degreeList</E>, i.e. for every vertex there is exactly one
 	entry of <E>degreeList</E> such that the vertex is incident to this
 	number of faces.
 	The <E>degreeList</E> is always sorted but may contain duplicates.

	 As an example, consider the five-star from the start of chapter
 	<Ref Chap="Chapter_Properties"/>:
 <Alt Only="TikZ">
   	\begin{tikzpicture}[vertexStyle, edgeStyle=nolabels, faceStyle]
   	   \def\r{2.5}

\coordinate (Z) at (0,0);
\foreach \i in {1,...,5}{
    \coordinate (P\i) at (90+\i*72:\r);
}

\draw[edge, face]
    (Z) -- (P4) -- node[edgeLabel]{6} (P5) -- cycle
    (Z) -- node[edgeLabel]{2} (P4) -- node[edgeLabel]{8} (P3) -- cycle
    (Z) -- node[edgeLabel]{3} (P3) -- node[edgeLabel]{9} (P2) -- cycle
    (Z) -- node[edgeLabel]{4} (P2) -- node[edgeLabel]{10} (P1) -- cycle
    (Z) -- node[edgeLabel]{5} (P1) -- node[edgeLabel]{7} (P5) -- node[edgeLabel]{1} cycle;

\foreach \p/\r/\n in {P1/left/11, P2/left/7, P3/right/5, P4/right/3, P5/above/2, Z/below/1}{
    \vertexLabelR{\p}{\r}{\n}
}

\foreach \p/\q/\n in {P1/P2/IV, P2/P3/III, P3/P4/II, P4/P5/I, P5/P1/V}{
    \node[faceLabel] at (barycentric cs:Z=1,\p=1,\q=1) {\n};
}

   	\end{tikzpicture}
 	</Alt>
 	@ExampleSession
 gap> countFac:=CounterOfFaces(fiveStar);
 counter of faces ([ [ 2, 2, 5 ] ] degrees, and [ 5 ] multiplicities) 
 gap> ListCounter(countFac);
 [ [ [ 2, 2, 5 ], 5 ] ]
 	@EndExampleSession
   </Description>
 </ManSection>

 <ManSection Label="CounterOfButterflies">
   <Oper Name="CounterOfButterflies" Arg="surface"
      Label="for IsSimplicialSurface"
      Comm="Construct a butterfly counter from a simplicial surface"/>
   <Returns>A Counter-&GAP;-object</Returns>
   <Prop Name="IsCounterOfButterflies" Arg="object" Label="for IsObject"
      Comm="Check whether a given object is a butterfly counter"/>
   <Returns><K>true</K> or <K>false</K></Returns>
   <Description>
       The method <K>CounterOfButterflies</K> constructs a new butterfly counter from
       a simplicial surface. The method <K>IsCounterOfButterflies</K>
       checks if a given &GAP;-object represents such a butterfly counter.
 	 The butterfly counter saves the information about how many butterflies have vertices with the same degrees.
       To get this information there are different possibilities.
       For example, the method <A>ListCounter</A> (<Ref Subsect="ListCounter"/>) that
       returns this information as a list of pairs <E>[[degList1,degList2], multiplicity]</E>, where
       <E>multiplicity</E> counts the number of butterflies whose vertex degrees match
       <E>[degList1,degList2]</E>, whereby <E>degList1</E> denotes the
       vertex degree of the vertices that are incident to the edge
       inducing the corresponding butterfly and <E>degList2</E> contains
       the vertex degrees of the two remaining vertices of the butterfly.
 
 	 As an example, consider the double-5-gon:
        <Alt Only="HTML">
 &lt;img src="./images/_Wrapper_Image_Double5gon-1.svg"> &lt;/img>
end{document}