                    \begin{tikzpicture}[vertexPlain=nolabels, edgeStyle=nolabels, faceStyle=nolabels]
                        % We need to define the drawing style
	
		        % First a tetrahedron
		        \begin{scope}[xshift=0cm]
			    \coordinate (A) at (0,0);
			    \coordinate (B) at (2,0);
		    	    \coordinate (C) at (0.8,1.5);
			    \coordinate (D) at (1.9,0.7);
			
                            \draw[face,edge]
                                (A) -- (B) -- (C) -- cycle
                                (B) -- (C) -- (D) -- cycle;
			    \draw[dashed,edge] (A) -- (D);
		        \end{scope}
		
		        % Second: four triangles in the form of a cone
		        \begin{scope}[xshift=3cm]
			    \coordinate (A) at (0,0);
			    \coordinate (B) at (1.7,0.5);
			    \coordinate (C) at (1.3,1.4);
			    \coordinate (D) at (0.5,1.5);
			    \coordinate (E) at (1,0.7);
			
			    % Take care to draw the faces in the back first
                            \draw[face,edge]
                                (A) -- (B) -- (C) -- cycle
                                (A) -- (C) -- (D) -- cycle
                                (A) -- (B) -- (E) -- cycle
                                (A) -- (E) -- (D) -- cycle;
                            \draw[edge, dashed] (A) -- (C);
		        \end{scope}
		
		        % Three triangles that share an edge
		        \begin{scope}[xshift=6cm]
                            \coordinate (A) at (0,0);
\coordinate (B) at (0,1.5);
\coordinate (C) at (-0.7,0.4);
\coordinate (D) at (0.8,0.4);
\coordinate (E) at (0.9,0.8);
			
\draw[edge, face]
    (A) -- (B) -- (E) -- cycle;
\draw[edge,face]
    (A) -- (B) -- (C) -- cycle
    (A) -- (B) -- (D) -- cycle;
\draw[edge, dashed] (A) -- (E);
		     

		        \end{scope}
		
		        % A butterfly of triangles
		        \begin{scope}[xshift=9cm]
                            \def\LUX{-0.8}
\def\LUY{-0.3}
\def\LMX{-1.2}
\def\LMY{0.5}
\def\LOX{-0.5}
\def\LOY{1}
\coordinate (A) at (0,0);
\coordinate (B) at (\LUX,\LUY);
\coordinate (C) at (\LMX,\LMY);
\coordinate (D) at (\LOX,\LOY);
\coordinate (E) at (-\LOX,\LOY);
\coordinate (F) at (-\LMX,\LMY);
\coordinate (G) at (-\LUX,\LUY);
			
\draw[face,edge]
    (A) -- (B) -- (C) -- cycle
    (A) -- (C) -- (D) -- cycle;
\draw[faceSwap, edge]
    (A) -- (E) -- (F) -- cycle
    (A) -- (F) -- (G) -- cycle;

\foreach \p/\r/\n in {A/below/1, D/above/2, C/left/3, B/below/4, G/below/5, F/right/6, E/above/7}{
    \vertexLabelR{\p}{\r}{\n}
}

\foreach \p/\q/\n in {B/C/III, C/D/I, F/G/IV, E/F/II}{
    \node[faceLabel] at (barycentric cs:A=1,\p=1,\q=1) {\n};
}

		        \end{scope}
		
		        % An open cone of two triangles
		        \begin{scope}[xshift=1cm, yshift=-3cm]
			    \coordinate (A) at (0,0);
			    \coordinate (B) at (1.3,0.4);
			    \coordinate (C) at (0.4,1.3);
			
                            \draw[face, edge]
                                (A) -- (B) to[bend right=45] (C) -- cycle
                                (A) -- (B) to[bend left=45] (C) -- cycle;
		        \end{scope}
		
		        % A surface from non-triangular shapes
		        \begin{scope}[xshift=4cm, yshift=-3cm]
			    \coordinate (A) at (0,0);
			    \coordinate (B) at (1,0);
			    \coordinate (C) at (0.5,0.6);
			    \coordinate (D) at (0,1);
			    \coordinate (E) at (0.5,1.3);
			    \coordinate (F) at (1,1);
			    \coordinate (G) at (1.7,0.8);
			    \coordinate (H) at (1.8,0.2);
			
                            \draw[face, edge]
                                (A) -- (B) -- (C) -- cycle
                                (A) -- (C) -- (D) -- cycle
                                (D) -- (C) -- (F) -- (E) -- cycle
                                (C) -- (F) -- (G) -- (H) -- (B) -- cycle;
		        \end{scope}

                        % Double tetrahedron
                        \begin{scope}[xshift=8.7cm, yshift=-2.3cm, scale=0.5]
                            % First tetrahedron is ABCD, second one is DEFG
\coordinate (A) at (-3,-1);
\coordinate (B) at (-1,-1.5);
\coordinate (C) at (-2,1);
\coordinate (D) at (0,0);
\coordinate (E) at (3,-0.1);
\coordinate (F) at (1.5,2);
\coordinate (G) at (1.2,1);

\filldraw[face] (A) -- (B) -- (C) -- cycle;
\filldraw[face] (B) -- (C) -- (D) -- cycle;
\filldraw[faceAlt] (D) -- (F) -- (E) -- cycle;

\draw[dashed] (A) -- (D);
\draw[dashed] (E) -- (G);
\draw[dashed] (D) -- (G);
\draw[dashed] (F) -- (G);

                        \end{scope}
                    \end{tikzpicture}
 
