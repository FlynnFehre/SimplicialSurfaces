\documentclass[11pt]{beamer}

% Allgemeine Definitionen
\usepackage[ngerman]{babel}	% Deutsche Sprachanpassung, z.B. Silbentrennung bei Worten mit Sonderzeichen
\usepackage[utf8]{inputenc}		% Direkte Eingabe von Umlauten

\usepackage{multimedia}		% Animations-Unterstützung (braucht hyperref | hyperref ist Teil von beamer )

% Verschiedene Operatoren
\newcommand{\union}{\ensuremath{\cup}}		% Die Vereinigung
\newcommand{\schnitt}{\ensuremath{\cap}}		% Der Schnitt
\newcommand{\und}{\ensuremath{\wedge}}		% Das und-Symbol
\newcommand{\oder}{\ensuremath{\vee}}		% Das oder-Symbol
\newcommand{\folgt}[1]{\ensuremath{\stackrel{#1}{\Rightarrow}}}	% Das Folgerungszeichen mit Symbol darüber 
\newcommand{\gleich}[1]{\ensuremath{\stackrel{#1}{=}}}	% Das Gleichheitszeichen mit Symbol darüber 
\newcommand{\isdef}{\ensuremath{\mathrel{\mathop:}=}}		% Das "ist-definiert"-Zeichen
\newcommand{\defis}{\ensuremath{=\mathrel{\mathop:}}}		% Das "ist-definiert"-Zeichen (andersrum)

% Weiterer nützlicher Kram
%\usepackage{enumitem}		% NICHT kompatibel mit beamer!
\usepackage{amsfonts, amsmath, amssymb}
\usepackage{amsthm}			% Eigene Umgebungen
\usepackage{thumbpdf}	% Seitenvorschau in PDF-Dokumenten

\usepackage{nicefrac}

\usepackage{pgf, tikz}
\usetikzlibrary{arrows, decorations.pathreplacing, calc}

% Erst die Farbe (optional), dann die Norm, dann der Inhalt
\newcommand{\norm}[3][black]{\ensuremath{\left\Vert #3\right\Vert_{\textcolor{#1}{#2}}}}

% Die folgenden Befehle dienen dazu, verschiedene Konventionen darzustellen. Man kann damit
% auf einfache Weise die in dem Dokument verwendeten Konventionen umschalten.

%\newcommand{\bisn}[1]{\ensuremath{\underline{#1}}}	% Bezeichnung für die Menge {1,..,n}
\newcommand{\bisn}[1]{\ensuremath{\{1,\dots , #1 \}}}	% Alternative (ohne Fallunterscheidung)

% Das Erzeugnis, das erste Argument zeigt an, welches Erzeugnis gemeint ist, das zweite gibt seinen Inhalt an.
%\newcommand{\spn}[2][]{\ensuremath{\mathrm{span} \{ #1 \}}}
\newcommand{\spn}[2][]{\ensuremath{\left\langle #2 \right\rangle _{ #1 }}}

\newcommand{\grad}[1]{\ensuremath{\mathrm{grad}( #1 )}}
%\newcommand{\grad}[1]{\ensuremath{\nabla #1}}

\newcommand{\tr}[1]{\ensuremath{#1^{tr}}} % Bezeichnung für die Transponierte einer Matrix

% Die Mengentheoretische Differenz
\newcommand{\ohne}{\ensuremath{-}}
%\newcommand{\ohne}{\ensuremath{\backslash}}

\DeclareMathOperator{\Hom}{Hom}
\DeclareMathOperator{\GL}{GL}
\DeclareMathOperator{\diag}{diag}
\DeclareMathOperator{\Rg}{Rang}
\DeclareMathOperator{\Bild}{Bild}
\DeclareMathOperator{\Kern}{Kern}
\DeclareMathOperator{\Spur}{Spur}
\DeclareMathOperator{\Aut}{Aut}
\DeclareMathOperator{\Sym}{Sym}
\DeclareMathOperator{\Pot}{Pot}
\DeclareMathOperator{\Grad}{Grad}
\DeclareMathOperator{\kgV}{kgV}

\newcommand{\menge}[1]{\ensuremath{\mathbb{#1}}}
\newcommand{\N}{\menge{N}}
\newcommand{\Z}{\menge{Z}}
\newcommand{\Q}{\menge{Q}}
\newcommand{\R}{\menge{R}}
\newcommand{\C}{\menge{C}}

% Mit den schönen Buchstaben arbeiten
\renewcommand{\phi}{\varphi}

\renewcommand{\subset}{\subseteq}

% Verschiedene mögliche Umgebungen
\newtheorem{lem}{Lemma}[section]		% Lemma
\newtheorem{bsp}[lem]{Beispiel}		% Beispiel
\newtheorem{folg}[lem]{Folgerung}	% Folgerung
\newtheorem{bem}[lem]{Bemerkung}
\newtheorem{satz}[lem]{Satz}			% Satz
\newtheorem{kor}[lem]{Korollar}		% Korollar
%\newtheorem{alg}[lem]{Algorithmus}	% Algorithmus
%\theoremstyle{definition}
\newtheorem{defi}[lem]{Definition}	% Definition
\theoremstyle{remark}
\newtheorem*{idea}{Beweisidee}	% Beweisidee (vor dem eigentlichen Beweis)

\beamertemplatenavigationsymbolsempty

\usetheme{Boadilla}
%	AnnArbor | Antibes | Bergen |
%	Berkeley | Berlin | Boadilla |
%	boxes | CambridgeUS | Copenhagen |
%	Darmstadt | default | Dresden |
%	Frankfurt | Goettingen |Hannover |
%	Ilmenau | JuanLesPins | Luebeck |
%	Madrid | Malmoe | Marburg |
%	Montpellier | PaloAlto | Pittsburgh |
%	Rochester | Singapore | Szeged |
%	Warsaw
%\usecolortheme{seahorse}
%	albatross | beaver | beetle |
%	crane | default | dolphin |
%	dove | fly | lily | orchid |
%	rose |seagull | seahorse |
%	sidebartab | structure |
%	whale | wolverine
%\usefonttheme{professionalfonts}
%	default | professionalfonts | serif |
%	structurebold | structureitalicserif |
%	structuresmallcapsserif
%\useinnertheme{rounded}
%	circles | default | inmargin |
%	rectangles | rounded
%\useoutertheme{shadow}
%	default | infolines | miniframes |
%	shadow | sidebar | smoothbars |
%	smoothtree | split | tree

% Graphic support
% This document contains the TikZ-header for all our LaTeX-computations.
% It especially contains all global graphic parameters.

\usepackage{amsmath, amssymb, amsfonts} % Standard Math-stuff

\usepackage{tikz}
\usetikzlibrary{calc}
\usetikzlibrary{positioning}

% Define a text=none option for nodes that ignores the given text, from
% https://tex.stackexchange.com/questions/59354/no-text-none-in-tikz
\makeatletter
\newif\iftikz@node@phantom
\tikzset{
  phantom/.is if=tikz@node@phantom,
  text/.code=%
    \edef\tikz@temp{#1}%
    \ifx\tikz@temp\tikz@nonetext
      \tikz@node@phantomtrue
    \else
      \tikz@node@phantomfalse
      \let\tikz@textcolor\tikz@temp
    \fi
}
\usepackage{etoolbox}
\patchcmd\tikz@fig@continue{\tikz@node@transformations}{%
  \iftikz@node@phantom
    \setbox\pgfnodeparttextbox\hbox{}
  \fi\tikz@node@transformations}{}{}
\makeatother

% Now we define the global styles
% The global styles are defined nestedly. You have to give your tikzpicture
% the global options [vertexStyle, edgeStyle, faceStyle] to activate them.
% 
% You can disable labels by using the option nolabels, i.e. 
% vertexStyle=nolabels to deactivate vertex labels.
%
% If you want to have a specific style for your picture, you can also use
% this specific meta-style instead of the general style. For example if you
% want to use double edges in one single picture - no matter the style of
% the rest of the document - you can use edgeDouble instead of edgeStyle.
%
% To set the default style, modify the vertexStyle/.default entry.

% Vertex styles
\tikzset{ 
    vertexNodePlain/.style = {fill=gray, shape=circle, inner sep=0pt, minimum size=2pt, text=none},
    vertexPlain/labels/.style = {
        vertexNode/.style={vertexNodePlain},
        vertexLabel/.style={gray}
    },
    vertexPlain/nolabels/.style = {
        vertexNode/.style={vertexNodePlain},
        vertexLabel/.style={text=none}
    },
    vertexPlain/.style = vertexPlain/#1,
    vertexPlain/.default=labels
}
\tikzset{
    vertexNodeNormal/.style = {fill=blue, shape=circle, inner sep=0pt, minimum size=4pt, text=none},
    vertexNormal/labels/.style = {
        vertexNode/.style={vertexNodeNormal},
        vertexLabel/.style={blue}
    },
    vertexNormal/nolabels/.style = {
        vertexNode/.style={vertexNodeNormal},
        vertexLabel/.style={text=none}
    },
    vertexNormal/.style = vertexNormal/#1,
    vertexNormal/.default=labels
}
\tikzset{
    vertexNodeBall/.style = {shape=circle, ball color=orange, inner sep=2pt, outer sep=0pt, minimum size=3pt},
    vertexBall/labels/.style = {
        vertexNode/.style={vertexNodeBall, text=black},
        vertexLabel/.style={text=none}
    },
    vertexBall/nolabels/.style = {
        vertexNode/.style={vertexNodeBall, text=none},
        vertexLabel/.style={text=none}
    },
    vertexBall/.style = vertexBall/#1,
    vertexBall/.default=labels
}
\tikzset{ 
    vertexStyle/.style={vertexNormal=#1},
    vertexStyle/.default = labels
}


% 1) position of the vertex
% 2) relative position of the node
% 3) name of the vertex
\newcommand{\vertexLabelR}[3]{
    \node[vertexLabel, #2] at (#1) {#3};
}
% 1) position of the vertex
% 2) absolute position of the node
% 3) name of the vertex
\newcommand{\vertexLabelA}[3]{
    \node[vertexLabel] at (#2) {#3};
}


% Edge styles
% If you have trouble with the double-lines overlapping, this might (?) help:
% https://tex.stackexchange.com/questions/288159/closing-the-ends-of-double-line-in-tikz
\tikzset{
    edgeLinePlain/.style={line join=round},
    edgePlain/labels/.style = {
        edge/.style={edgeLinePlain},
        edgeLabel/.style={fill=blue!20!white}
    },
    edgePlain/nolabels/.style = {
        edge/.style={edgeLinePlain},
        edgeLabel/.style={text=none}
    },
    edgePlain/.style = edgePlain/#1,
    edgePlain/.default = labels
}
\tikzset{
    edgeLineDouble/.style = {thin, double=gray!90!white, double distance=.3pt, line join=round},
    edgeDouble/labels/.style = {
        edge/.style = {edgeLineDouble},
        edgeLabel/.style = {fill=blue!20!white}
    },
    edgeDouble/nolabels/.style = {
        edge/.style = {edgeLineDouble},
        edgeLabel/.style = {text=none}
    },
    edgeDouble/.style = edgeDouble/#1,
    edgeDouble/.default = labels
}
\tikzset{
    edgeStyle/.style = {edgePlain=#1},
    edgeStyle/.default = labels
}

% Face styles
% Here we have an exception - the style face is always defined.
% 
\newcommand{\faceColorY}{yellow!60!white}   % yellow
\newcommand{\faceColorB}{blue!60!white}     % blue
\newcommand{\faceColorP}{cyan!60}           % purple
\newcommand{\faceColorR}{red!60!white}      % red
\newcommand{\faceColorG}{green!60!white}    % green
\newcommand{\faceColorO}{orange!50!yellow!70!white} % orange

\newcommand{\faceColor}{\faceColorY}
\newcommand{\faceColorSwap}{\faceColorB}
\tikzset{
    face/.style = {fill=#1},
    face/.default = \faceColor,
    faceY/.style = {face=\faceColorY},
    faceB/.style = {face=\faceColorB},
    faceP/.style = {face=\faceColorP},
    faceR/.style = {face=\faceColorR},
    faceG/.style = {face=\faceColorG},
    faceO/.style = {face=\faceColorO}
}
\tikzset{
    faceStyle/labels/.style = {
        faceNode/.style = {}
    },
    faceStyle/nolabels/.style = {
        faceNode/.style = {text=none}
    },
    faceStyle/.style = faceStyle/#1,
    faceStyle/.default = labels
}
\tikzset{ face/.style={fill=#1} }
\tikzset{ faceSwap/.code=
    \ifdefined\swapColours
        {face=\faceColorSwap}
    \else
        {face=\faceColor}
    \fi
}

% Sometimes we want to implement different behaviour for the generated 
% HTML-pictures (for example, shading is not supported in HTML).
% For that we define a macro to check whether we run the code with
% htlatex. The code comes from 
% https://tex.stackexchange.com/questions/93852/what-is-the-correct-way-to-check-for-latex-pdflatex-and-html-in-the-same-latex
\makeatletter
\edef\texforht{TT\noexpand\fi
  \@ifpackageloaded{tex4ht}
    {\noexpand\iftrue}
    {\noexpand\iffalse}}
\makeatother


\usepackage{hyperref}




\author{Markus Baumeister}
\title{Simplicial surfaces in GAP}
\date{??.08.2017}

\begin{document}

% Titelseite
\begin{frame}
\titlepage
\end{frame}


\begin{frame}
    \tableofcontents
\end{frame}

\section{General polygonal complexes by incidence geometry}
\begin{frame}
    \frametitle{Motivation}
    \pause
    Goal: simplicial surfaces (and generalisations) in GAP
    \pause
    \begin{center}
                            \begin{tikzpicture}[vertexPlain=nolabels, edgeStyle=nolabels, faceStyle=nolabels]
                        % We need to define the drawing style
	
		        % First a tetrahedron
		        \begin{scope}[xshift=0cm]
			    \coordinate (A) at (0,0);
			    \coordinate (B) at (2,0);
		    	    \coordinate (C) at (0.8,1.5);
			    \coordinate (D) at (1.9,0.7);
			
                            \draw[face,edge]
                                (A) -- (B) -- (C) -- cycle
                                (B) -- (C) -- (D) -- cycle;
			    \draw[dashed,edge] (A) -- (D);
		        \end{scope}
		
		        % Second: four triangles in the form of a cone
		        \begin{scope}[xshift=3cm]
                            			    \coordinate (A) at (0,0);
			    \coordinate (B) at (1.7,0.5);
			    \coordinate (C) at (1.3,1.4);
			    \coordinate (D) at (0.5,1.5);
			    \coordinate (E) at (1,0.7);
			
			    % Take care to draw the faces in the back first
                            \draw[face,edge]
                                (A) -- (B) -- (C) -- cycle
                                (A) -- (C) -- (D) -- cycle;
                            \draw[face,edge]
                                (A) -- (B) -- (E) -- cycle
                                (A) -- (E) -- (D) -- cycle;
                            \draw[edge, dashed] (A) -- (C);

		        \end{scope}
		
		        % Three triangles that share an edge
		        \begin{scope}[xshift=7cm]
                            \coordinate (A) at (0,0);
\coordinate (B) at (0,1.5);
\coordinate (C) at (-0.7,0.4);
\coordinate (D) at (0.8,0.4);
\coordinate (E) at (0.9,0.8);
			
\draw[edge, face]
    (A) -- (B) -- (E) -- cycle;
\draw[edge,face]
    (A) -- (B) -- (C) -- cycle
    (A) -- (B) -- (D) -- cycle;
\draw[edge, dashed] (A) -- (E);
		     

		        \end{scope}
		
		        % A butterfly of triangles
		        \begin{scope}[xshift=10cm]
                            \def\LUX{-0.8}
\def\LUY{-0.3}
\def\LMX{-1.2}
\def\LMY{0.5}
\def\LOX{-0.5}
\def\LOY{1}
\coordinate (A) at (0,0);
\coordinate (B) at (\LUX,\LUY);
\coordinate (C) at (\LMX,\LMY);
\coordinate (D) at (\LOX,\LOY);
\coordinate (E) at (-\LOX,\LOY);
\coordinate (F) at (-\LMX,\LMY);
\coordinate (G) at (-\LUX,\LUY);
			
\draw[face,edge]
    (A) -- (B) -- (C) -- cycle
    (A) -- (C) -- (D) -- cycle;
\draw[faceSwap, edge]
    (A) -- (E) -- (F) -- cycle
    (A) -- (F) -- (G) -- cycle;

\foreach \p/\r/\n in {A/below/1, D/above/2, C/left/3, B/below/4, G/below/5, F/right/6, E/above/7}{
    \vertexLabelR{\p}{\r}{\n}
}

\foreach \p/\q/\n in {B/C/III, C/D/I, F/G/IV, E/F/II}{
    \node[faceLabel] at (barycentric cs:A=1,\p=1,\q=1) {\n};
}

		        \end{scope}
		
		        % An open cone of two triangles
		        \begin{scope}[xshift=1cm, yshift=-3cm]
			    \coordinate (A) at (0,0);
			    \coordinate (B) at (1.3,0.4);
			    \coordinate (C) at (0.4,1.3);
			
                            \draw[face, edge]
                                (A) -- (B) to[bend right=45] (C) -- cycle
                                (A) -- (B) to[bend left=45] (C) -- cycle;
		        \end{scope}
		
		        % A surface from non-triangular shapes
		        \begin{scope}[xshift=4cm, yshift=-3cm]
			    \coordinate (A) at (0,0);
			    \coordinate (B) at (1,0);
			    \coordinate (C) at (0.5,0.6);
			    \coordinate (D) at (0,1);
			    \coordinate (E) at (0.5,1.3);
			    \coordinate (F) at (1,1);
			    \coordinate (G) at (1.7,0.8);
			    \coordinate (H) at (1.8,0.2);
			
                            \draw[face, edge]
                                (A) -- (B) -- (C) -- cycle
                                (A) -- (C) -- (D) -- cycle
                                (D) -- (C) -- (F) -- (E) -- cycle
                                (C) -- (F) -- (G) -- (H) -- (B) -- cycle;
		        \end{scope}

                        % Double tetrahedron
                        \begin{scope}[xshift=8.7cm, yshift=-2.3cm, scale=0.5]
                            % First tetrahedron is ABCD, second one is DEFG
\coordinate (A) at (-3,-1);
\coordinate (B) at (-1,-1.5);
\coordinate (C) at (-2,1);
\coordinate (D) at (0,0);
\coordinate (E) at (3,-0.1);
\coordinate (F) at (1.5,2);
\coordinate (G) at (1.2,1);

\filldraw[face] (A) -- (B) -- (C) -- cycle;
\filldraw[face] (B) -- (C) -- (D) -- cycle;
\filldraw[faceAlt] (D) -- (F) -- (E) -- cycle;

\draw[dashed] (A) -- (D);
\draw[dashed] (E) -- (G);
\draw[dashed] (D) -- (G);
\draw[dashed] (F) -- (G);

                        \end{scope}
                    \end{tikzpicture}
 

    \end{center}
\end{frame}

\section{Edge colouring and group properties}


\section{Abstract folding}


\end{document}
