% Allgemeine Definitionen
\usepackage[english]{babel}	% Deutsche Sprachanpassung, z.B. Silbentrennung bei Worten mit Sonderzeichen
\usepackage[utf8]{inputenc}		% Direkte Eingabe von Umlauten

\usepackage{multimedia}		% Animations-Unterstützung (braucht hyperref | hyperref ist Teil von beamer )

% Verschiedene Operatoren
\newcommand{\union}{\ensuremath{\cup}}		% Die Vereinigung
\newcommand{\schnitt}{\ensuremath{\cap}}		% Der Schnitt
\newcommand{\und}{\ensuremath{\wedge}}		% Das und-Symbol
\newcommand{\oder}{\ensuremath{\vee}}		% Das oder-Symbol
\newcommand{\folgt}[1]{\ensuremath{\stackrel{#1}{\Rightarrow}}}	% Das Folgerungszeichen mit Symbol darüber 
\newcommand{\gleich}[1]{\ensuremath{\stackrel{#1}{=}}}	% Das Gleichheitszeichen mit Symbol darüber 
\newcommand{\isdef}{\ensuremath{\mathrel{\mathop:}=}}		% Das "ist-definiert"-Zeichen
\newcommand{\defis}{\ensuremath{=\mathrel{\mathop:}}}		% Das "ist-definiert"-Zeichen (andersrum)

% Weiterer nützlicher Kram
%\usepackage{enumitem}		% NICHT kompatibel mit beamer!
\usepackage{amsfonts, amsmath, amssymb}
\usepackage{amsthm}			% Eigene Umgebungen
\usepackage{thumbpdf}	% Seitenvorschau in PDF-Dokumenten

\usepackage{nicefrac}

\usepackage{pgf, tikz}
\usetikzlibrary{arrows, decorations.pathreplacing, calc, decorations.pathmorphing, shapes}

% Erst die Farbe (optional), dann die Norm, dann der Inhalt
\newcommand{\norm}[3][black]{\ensuremath{\left\Vert #3\right\Vert_{\textcolor{#1}{#2}}}}

% Die folgenden Befehle dienen dazu, verschiedene Konventionen darzustellen. Man kann damit
% auf einfache Weise die in dem Dokument verwendeten Konventionen umschalten.

%\newcommand{\bisn}[1]{\ensuremath{\underline{#1}}}	% Bezeichnung für die Menge {1,..,n}
\newcommand{\bisn}[1]{\ensuremath{\{1,\dots , #1 \}}}	% Alternative (ohne Fallunterscheidung)

% Das Erzeugnis, das erste Argument zeigt an, welches Erzeugnis gemeint ist, das zweite gibt seinen Inhalt an.
%\newcommand{\spn}[2][]{\ensuremath{\mathrm{span} \{ #1 \}}}
\newcommand{\spn}[2][]{\ensuremath{\left\langle #2 \right\rangle _{ #1 }}}

\newcommand{\grad}[1]{\ensuremath{\mathrm{grad}( #1 )}}
%\newcommand{\grad}[1]{\ensuremath{\nabla #1}}

\newcommand{\tr}[1]{\ensuremath{#1^{tr}}} % Bezeichnung für die Transponierte einer Matrix

% Die Mengentheoretische Differenz
\newcommand{\ohne}{\ensuremath{-}}
%\newcommand{\ohne}{\ensuremath{\backslash}}

\DeclareMathOperator{\Hom}{Hom}
\DeclareMathOperator{\GL}{GL}
\DeclareMathOperator{\diag}{diag}
\DeclareMathOperator{\Rg}{Rang}
\DeclareMathOperator{\Bild}{Bild}
\DeclareMathOperator{\Kern}{Kern}
\DeclareMathOperator{\Spur}{Spur}
\DeclareMathOperator{\Aut}{Aut}
\DeclareMathOperator{\Sym}{Sym}
\DeclareMathOperator{\Pot}{Pot}
\DeclareMathOperator{\Grad}{Grad}
\DeclareMathOperator{\kgV}{kgV}

\newcommand{\menge}[1]{\ensuremath{\mathbb{#1}}}
\newcommand{\N}{\menge{N}}
\newcommand{\Z}{\menge{Z}}
\newcommand{\Q}{\menge{Q}}
\newcommand{\R}{\menge{R}}
\newcommand{\C}{\menge{C}}

% Mit den schönen Buchstaben arbeiten
\renewcommand{\phi}{\varphi}

\renewcommand{\subset}{\subseteq}

% Verschiedene mögliche Umgebungen
\newtheorem{lem}{Lemma}[section]		% Lemma
\newtheorem{bsp}[lem]{Beispiel}		% Beispiel
\newtheorem{folg}[lem]{Folgerung}	% Folgerung
\newtheorem{bem}[lem]{Bemerkung}
\newtheorem{satz}[lem]{Satz}			% Satz
\newtheorem{kor}[lem]{Korollar}		% Korollar
%\newtheorem{alg}[lem]{Algorithmus}	% Algorithmus
%\theoremstyle{definition}
\newtheorem{defi}[lem]{Definition}	% Definition
\theoremstyle{remark}
\newtheorem*{idea}{Beweisidee}	% Beweisidee (vor dem eigentlichen Beweis)


