\section{Abstract folding}
\frame{\tableofcontents[currentsection]}

\begin{frame}{What kind of folding?}
    \pause
    There are many different kinds of folding (e.\,g. Origami)
    \pause
    Here:
    \begin{itemize}
        \pause
        \item Folding of surface in $\R^3$
        \pause
        \item Possible folding edges are fixed
        \pause
        \item Folding should be rigid (no curvature)
    \end{itemize}

    \pause
    Goal: Classify possible folding patterns (given a net)

    \pause
    \begin{center}
        \movie[ height = 0.4\textwidth, width = 0.4\textwidth, autostart, loop ]{}{TorusNetz.mp4}
    \end{center}

\end{frame}


\begin{frame}{Embeddings}
    \pause
    Ideally, we would like to have embeddings.
    
    \pause
    But we want to define folding independently from an embedding, since:

    \begin{itemize}
        \pause
        \item They are very hard to compute (even for small examples)
        \pause
        \item We can only show foldability for specific small examples
            \begin{itemize}
                \pause
                \item Usually using regularity (like crystallographic symmetry)
                \pause
                \item No general method
            \end{itemize}
        \pause
        \item It is very hard to define iterated folding in an embedding
    \end{itemize}

    \pause
    \begin{center}
        \movie[ height = 0.4\textwidth, width = 0.4\textwidth, autostart, loop ]{}{TorusNetz.mp4}
    \end{center}

\end{frame}


\begin{frame}{What is folding without embedding?}
    \pause
    Central idea:
    \begin{itemize}
        \pause
        \item Don't model folding process
        \pause
        \item Describe result of folding process
    \end{itemize}

    \pause
    $\leadsto$ Only consider changes in the topology 
    \pause (like identification of faces)
    \newline

    \pause
    Solution: Incidence geometry (polygonal complex/surface)
    \begin{itemize}
        \pause
        \item Ignores angles and lengths (mostly)
        \pause
        \item Keeps folding restrictions (tetrahedron is rigid)
        \pause
        \item Allows an abstract definition of folding
    \end{itemize}

\end{frame}



