\section{Abstract folding}
\frame{\tableofcontents[currentsection]}

\begin{frame}{What kind of folding?}
    \pause
    There are many different kinds of folding (e.\,g. Origami)
    \pause
    Here:
    \begin{itemize}
        \pause
        \item Folding of surface in $\R^3$
        \pause
        \item Possible folding edges are fixed
        \pause
        \item Folding should be rigid (no curvature)
    \end{itemize}

    \pause
    Goal: Classify possible folding patterns (given a net)

    \pause
    \begin{center}
        \movie[ height = 0.4\textwidth, width = 0.4\textwidth, autostart, loop ]{}{TorusNetz.mp4}
    \end{center}

\end{frame}


\begin{frame}{Why are embeddings hard?}
    \pause
    Ideally, we would like to have embeddings.
    
    \pause
    But we want to define folding independently from an embedding, since:

    \begin{itemize}
        \pause
        \item They are very hard to compute (even for small examples)
        \pause
        \item We can only show foldability for specific small examples
            \begin{itemize}
                \pause
                \item Usually using regularity (like crystallographic symmetry)
                \pause
                \item No general method
            \end{itemize}
        \pause
        \item It is very hard to define iterated folding in an embedding
    \end{itemize}

    \pause
    \begin{center}
        \movie[ height = 0.4\textwidth, width = 0.4\textwidth, autostart, loop ]{}{TorusNetz.mp4}
    \end{center}

\end{frame}

%TODO what about forced folding? Or closedness of surfaces under folding? RETHINK this

\begin{frame}{Is there an alternative?}
    \pause
    Central idea:
    \begin{itemize}
        \pause
        \item Don't model folding process (needs embedding)
        \pause
        \item Describe starting and final folding state
            \begin{itemize}
                \pause
                \item Only consider changes in the topology
                    \pause (like identification of faces)
                \pause
                \item allows abstraction from embedding
            \end{itemize}
    \end{itemize}

    % Use pentagon-flippy to illustrate incidence rigidity

    \pause
    $\leadsto$ Incidence geometry (polygonal complex/surface)

    \begin{itemize}
        \pause
        \item Captures some folding restrictions \pause (rigidity of tetrahedron)
        \pause
        \item Still needs a lot of refinement
    \end{itemize}
\end{frame}


\begin{frame}{Important properties of folding}
    \begin{itemize}
        \item<2-> The class of surfaces is not closed under folding
        \item<3-> Folding can be undone by \textit{unfolding}
        \item<4-> Identification of two faces might force identification of two other faces
            \begin{itemize}
                \item 
            \end{itemize}
    \end{itemize}

    \begin{center}
        \begin{tikzpicture}
            % First identification
            \only<5-6>{
                \def\Hdist{1.5}
		\def\Vdist{1}
			
		\coordinate (A) at (-\Hdist,0);
		\coordinate (B) at (0,\Vdist);
		\coordinate (C) at (\Hdist,0);
		\coordinate (D) at (0,-\Vdist);
		\foreach \p in {A,B,C,D}
		    \fill [black] (\p) circle (1.5pt);
		\draw [<->,red,thick] ($(B)!0.5!(A)$) -- ($(D)!0.5!(A)$);
                \uncover<6>{
                    \draw[<->,red,thick] ($(B)!0.5!(C)$) -- ($(D)!0.5!(C)$);
                }
            }
        \end{tikzpicture}
    \end{center}
\end{frame}


