% This document contains the TikZ-header for all our LaTeX-computations.
% It especially contains all global graphic parameters.

\usepackage{amsmath, amssymb, amsfonts} % Standard Math-stuff

\usepackage{tikz}
\usetikzlibrary{calc}
\usetikzlibrary{positioning}


% Now we define the global styles

% We start by defining the default colours of vertices, edges and faces
\newcommand{\vertexColor}{\colorVertex}
\newcommand{\edgeColor}{\colorEdge}
\newcommand{\faceColor}{\colorFace}
% If we want to shade a surface with two colours, we need an alternative
\newcommand{\faceColorDark}{\colorDarkFace}

\newcommand{\vSize}{2pt}    % How big are the vertex circles (if drawn)?


\tikzset{vertex/.style = {\vertexColor}}
\tikzset{edge/.style = {\edgeColor, thick}}
\tikzset{edgeBackground/.style = {fill=\colorEdgeBack}}
\tikzset{face/.style = {fill=\faceColor, draw=\edgeColor}}
% The face style for the alternative colouring. The colour changes if
% \differentColors is defined.
\tikzset{faceDark/.style = {fill=\faceColorDark, draw=\edgeColor}}


% Custom command to draw an edge between two points
% First two parameters are the names of the points (no brackets)
% Third parameter is the name of the edge
\newcommand{\drawEdge}[3]{
    \node[edgeBackground] at ($1/2*(#1)+1/2*(#2)$) {#3};
}

% Custom command to draw a bi-arrow with a "knick" TODO
% First parameter: Draw options
% Second parameter: Start point
% Third parameter: Middle point
% Fourth parameter: Final point
\newcommand{\planEdge}[4]{
    \draw[#1,<-] #2 -- #3;
    \draw[#1,->] #3 -- #4;
}

% Draw a plane behind the second and third parameters (defined by first one)
\newcommand{\behindPlane}[3]{
    \filldraw[fill=black!20!white,draw=black!50!white] 
        (#2) -- (#3) -- ($(#3)+(#1)$) -- ($(#2)+(#1)$) -- cycle;
    \draw (#2) -- (#3);
}

% Sometimes we want to implement different behaviour for the generated 
% HTML-pictures (for example, shading is not supported in HTML).
% For that we define a macro to check whether we run the code with
% htlatex. The code comes from 
% https://tex.stackexchange.com/questions/93852/what-is-the-correct-way-to-check-for-latex-pdflatex-and-html-in-the-same-latex
\makeatletter
\edef\texforht{TT\noexpand\fi
  \@ifpackageloaded{tex4ht}
    {\noexpand\iftrue}
    {\noexpand\iffalse}}
\makeatother


\usepackage{hyperref}
