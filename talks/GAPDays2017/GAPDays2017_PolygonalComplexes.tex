\section{General simplicial surfaces}
\frame{\tableofcontents[currentsection]}

\begin{frame}{\uncover<3->{Triangular complexes}}
    We want to describe different structures:
    \uncover<2->{
        \begin{center}
            \begin{tikzpicture}[scale=2]
    \begin{scope}[scale=0.5,yshift=0.8cm]
                    % Coordinates of octahedron
                \def\len{2} % The length of the base
                \def\h{0.9}   % \h*\len is the hight of the apex above the 
                            % center of the square
                \coordinate (Mid) at (0,0);
                \coordinate (Right) at (0.5*\len,0.3*\len);
                \coordinate (Left) at (-0.75*\len,0.25*\len);
                \coordinate (Back) at ($(Right)+(Left)$);
                \coordinate (Center) at ($(Left)!0.5!(Right)$);
                \coordinate (Up) at ($(Center)+(0,\h*\len)$);
                \coordinate (Down) at ($(Center)+(0,-\h*\len)$);

                \filldraw[face] (Up) -- (Mid) -- (Left) -- cycle;
                \filldraw[face] (Up) -- (Mid) -- (Right) -- cycle;
                \filldraw[face] (Down) -- (Mid) -- (Left) -- cycle;
                \filldraw[face] (Down) -- (Mid) -- (Right) -- cycle;
                \draw[dashed] (Up) -- (Back) -- (Down);
                \draw[dashed] (Left) -- (Back) -- (Right);

 

    \end{scope}
    
    \begin{scope}[xshift=1cm]
        			    \coordinate (A) at (0,0);
			    \coordinate (B) at (1.7,0.5);
			    \coordinate (C) at (1.3,1.4);
			    \coordinate (D) at (0.5,1.5);
			    \coordinate (E) at (1,0.7);
			
			    % Take care to draw the faces in the back first
                            \draw[face,edge]
                                (A) -- (B) -- (C) -- cycle
                                (A) -- (C) -- (D) -- cycle;
                            \draw[face,edge]
                                (A) -- (B) -- (E) -- cycle
                                (A) -- (E) -- (D) -- cycle;
                            \draw[edge, dashed] (A) -- (C);

    \end{scope}

    \begin{scope}[xshift=3.5cm]
        			    \coordinate (A) at (0,0);
			    \coordinate (B) at (1.3,0.4);
			    \coordinate (C) at (0.4,1.3);
			
			    \filldraw[face] (A) -- (B) to[bend right=45] (C) -- cycle;
			    \filldraw[face] (A) -- (B) to[bend left=45] (C) -- cycle;


    \end{scope}
\end{tikzpicture}


        \end{center}
    }
    \uncover<3->{
        $\leadsto$ \textbf{triangular complexes}
    }
    \begin{itemize}
        \item<4-> sets of vertices, edges and faces
        \item<5-> incidence relation between them
        \item<6-> every face is a triangle
        \item<7-> every vertex lies in an edge \uncover<8->{and every edge lies in a face}
    \end{itemize}
\end{frame}


\begin{frame}{Isomorphism testing}
    Incidence structure can be interpreted as a coloured graph:
    \begin{center}
        \begin{tikzpicture}[vertexNode/.style={circle,fill=\colorVertexNode},node distance=0.5,
                faceNode/.style={fill=\faceColor,regular polygon, regular polygon sides=3}]
            \def\len{1.5}
            
            \uncover<4->{
                \coordinate[label={[vertex]left:5}] (A) at (0,0);
                \coordinate[label={[vertex]above:6}] (B) at (\len,\len);
                \coordinate[label={[vertex]below:7}] (C) at (\len,-\len);
            }
            \uncover<7->{
                \coordinate[label={[vertex]right:11}] (D) at (2*\len,0);
            }

            \uncover<2->{
                \filldraw[face] (A) -- (B) -- (C) -- cycle;
                \node at (barycentric cs:A=1,B=1,C=1) {1};
            }
            \uncover<5->{
                \filldraw[face] (B) -- (C) -- (D) -- cycle;
                \node at (barycentric cs:B=1,C=1,D=1) {8};
            }

            \uncover<3->{
                \drawEdge{A}{B}{2}
                \drawEdge{A}{C}{3}
                \drawEdge{B}{C}{4}
            }
            \uncover<6->{
                \drawEdge{B}{D}{9}
                \drawEdge{C}{D}{10}
            }

            \uncover<4->{
                \fill[vertex] (A) circle (\vSize);
                \fill[vertex] (B) circle (\vSize);
                \fill[vertex] (C) circle (\vSize);
            }
            \uncover<7->{
                \fill[vertex] (D) circle (\vSize);
            }


            \begin{scope}[xshift=5cm]
                \uncover<3->{
                    \node[edgeBackground] (E1) {2};
                    \node[edgeBackground] (E2) [right=of E1] {3};
                    \node[edgeBackground] (E3) [right=of E2] {4};
                }
                \uncover<6->{
                    \node[edgeBackground] (E4) [right=of E3] {9};
                    \node[edgeBackground] (E5) [right=of E4] {10};
                }

                \uncover<2->{
                    \node[faceNode] (F1) [above=of E2] {1}; 
                }
                \uncover<5->{
                    \node[faceNode] (F2) [above=of E4] {8};
                }

                \uncover<4->{
                    \node[vertexNode] (V1) [below=of E1] {5}
                        edge (E1) edge (E2);
                    \node[vertexNode] (V2) [right=of V1] {6}
                        edge (E1) edge (E3);
                    \node[vertexNode] (V3) [right=of V2] {7}
                        edge (E2) edge (E3);
                }
                \uncover<7->{
                    \node[vertexNode] (V4) [right=of V3] {11}
                        edge (E4) edge (E5);
                }

                \uncover<3->{
                    \draw (F1) -- (E1);
                    \draw (F1) -- (E2);
                    \draw (F1) -- (E3);
                }

                \uncover<5->{
                    \draw (F2) -- (E3);
                }

                \uncover<6->{
                    \draw (F2) -- (E4) -- (V2);
                    \draw (F2) -- (E5) -- (V3);
                }
            \end{scope}
        \end{tikzpicture}
    \end{center}

    \uncover<8->{
        $\leadsto$ reduce to graph isomorphism problem
    }

    \uncover<9->{
        $\leadsto$ can be solved quite easily by \texttt{Nauty} (McKay, Piperno)
    }

    \uncover<10->{
        Interfaced by NautyTracesInterface (by Gutsche, Niemeyer, Schweitzer)
        \begin{itemize}
            \item<11-> direct C--interface without writing files
            \item<12-> also returns automorphism group
        \end{itemize}
    }
\end{frame}


\begin{frame}{General properties}
    \pause
    Some properties can be computed for all triangular complexes:
    \pause
    \begin{itemize}
        \item Connectivity
        \pause
        \item Euler--Characteristic
    \end{itemize}
    \pause
    \textit{Orientability} is \textbf{not} one of them.
    \pause
    Counterexample:
        \begin{center}
            \begin{tikzpicture}
                \coordinate (A) at (0,0);
\coordinate (B) at (0,1.5);
\coordinate (C) at (-0.7,0.4);
\coordinate (D) at (0.8,0.4);
\coordinate (E) at (0.9,0.8);
			
\draw[edge, face]
    (A) -- (B) -- (E) -- cycle;
\draw[edge,face]
    (A) -- (B) -- (C) -- cycle
    (A) -- (B) -- (D) -- cycle;
\draw[edge, dashed] (A) -- (E);
		     

	    \end{tikzpicture}
        \end{center}
    \pause
    $\Rightarrow$ every edge lies in at most two faces (for well--definedness)
    
    \pause
    $\leadsto$ \textbf{ramified simplicial surfaces}
\end{frame}


\begin{frame}{Why ramified?}
    \pause
    Typical example of ramified simplicial surface:
    \pause
    \begin{center}
        \begin{tikzpicture}
            \begin{tikzpicture}[vertexBall, edgeDouble, faceStyle, scale=2]

% Define the coordinates of the vertices
\coordinate (V1_1) at (0., 0.);
\coordinate (V2_1) at (1., 0.);
\coordinate (V3_1) at (-0.5, 0.8660254037844384);
\coordinate (V3_2) at (1.5, 0.8660254037844388);
\coordinate (V4_1) at (0.4999999999999999, 0.8660254037844386);
\coordinate (V5_1) at (0.5000000000000001, -0.8660254037844386);
\coordinate (V5_2) at (-0.9999999999999996, 0.);
\coordinate (V5_3) at (2., 0.);


% Fill in the faces
\fill[face]  (V2_1) -- (V4_1) -- (V1_1) -- cycle;
\node[faceLabel] at (barycentric cs:V2_1=1,V4_1=1,V1_1=1) {$1$};
\fill[face]  (V2_1) -- (V3_2) -- (V4_1) -- cycle;
\node[faceLabel] at (barycentric cs:V2_1=1,V3_2=1,V4_1=1) {$2$};
\fill[face]  (V4_1) -- (V3_1) -- (V1_1) -- cycle;
\node[faceLabel] at (barycentric cs:V4_1=1,V3_1=1,V1_1=1) {$3$};
\fill[face]  (V1_1) -- (V5_1) -- (V2_1) -- cycle;
\node[faceLabel] at (barycentric cs:V1_1=1,V5_1=1,V2_1=1) {$4$};
\fill[face]  (V3_1) -- (V5_2) -- (V1_1) -- cycle;
\node[faceLabel] at (barycentric cs:V3_1=1,V5_2=1,V1_1=1) {$5$};
\fill[face]  (V2_1) -- (V5_3) -- (V3_2) -- cycle;
\node[faceLabel] at (barycentric cs:V2_1=1,V5_3=1,V3_2=1) {$6$};


% Draw the edges
\draw[edge] (V2_1) -- node[edgeLabel] {$1$} (V1_1);
\draw[edge] (V1_1) -- node[edgeLabel] {$2$} (V3_1);
\draw[edge] (V1_1) -- node[edgeLabel] {$3$} (V4_1);
\draw[edge] (V5_1) -- node[edgeLabel] {$4$} (V1_1);
\draw[edge] (V1_1) -- node[edgeLabel] {$4$} (V5_2);
\draw[edge] (V3_2) -- node[edgeLabel] {$5$} (V2_1);
\draw[edge] (V4_1) -- node[edgeLabel] {$6$} (V2_1);
\draw[edge] (V2_1) -- node[edgeLabel] {$7$} (V5_1);
\draw[edge] (V5_3) -- node[edgeLabel] {$7$} (V2_1);
\draw[edge] (V3_1) -- node[edgeLabel] {$8$} (V4_1);
\draw[edge] (V4_1) -- node[edgeLabel] {$8$} (V3_2);
\draw[edge] (V5_2) -- node[edgeLabel] {$9$} (V3_1);
\draw[edge] (V3_2) -- node[edgeLabel] {$9$} (V5_3);


% Draw the vertices
\vertexLabelR{V1_1}{left}{$1$}
\vertexLabelR{V2_1}{left}{$2$}
\vertexLabelR{V3_1}{left}{$3$}
\vertexLabelR{V3_2}{left}{$3$}
\vertexLabelR{V4_1}{left}{$4$}
\vertexLabelR{V5_1}{left}{$5$}
\vertexLabelR{V5_2}{left}{$5$}
\vertexLabelR{V5_3}{left}{$5$}

\end{tikzpicture}

        \end{tikzpicture}
    \end{center}
    \pause
    $\Rightarrow$ It is not a surface -- there is a \textit{ramification} at 
        the central vertex
    
    \pause
    A \textbf{simplicial surface} does not have these ramifications.
\end{frame}


\begin{frame}{Classification}
    \pause
    Plesken/Strzelczyk classified all closed simplicial surfaces up to 20 triangles.

    \begin{itemize}
        \pause
        \item only interesting for those without a 3--cycle of edges
        \pause
        \item e.\,g. there are 87 non--isomorphic surfaces with 20 triangles
        \pause
        \item e.\,g. there is only one surface with 10 triangles:
    \end{itemize}

    \pause
    \begin{center}
        \begin{tikzpicture}[z={(0cm,1cm)},x={(-1cm,-1cm)},y={(1cm,-1cm)}, faceOp/.style={opacity=0.7}]
            \def\len{1.5}
            \def\h{0.5}

            \coordinate (Top) at (0,0,\h);
            \coordinate (Bottom) at (0,0,-\h);
            \foreach \i in {0,...,4}{
                \coordinate (P\i) at (-10+72*\i:\len);
            }

            \foreach \i in {0,...,4}{
                \draw[dashed] (Bottom) -- (P\i);
            }

            \filldraw[face,faceOp] (Top) -- (P3) -- (P4) -- cycle;
            \filldraw[face,faceOp] (Top) -- (P2) -- (P3) -- cycle;
            \filldraw[face,faceOp] (Top) -- (P4) -- (P0) -- cycle;
            \filldraw[face,faceOp] (Top) -- (P1) -- (P2) -- cycle;
            \filldraw[face,faceOp] (Top) -- (P0) -- (P1) -- cycle;

        \end{tikzpicture}
    \end{center}
\end{frame}


\begin{frame}{Progress report of triangulated complexes}
    \pause
    Already implemented:
    \begin{itemize}
        \pause
        \item surface hierarchy
        \pause
        \item elementary properties (e.\,g. connectivity, orientability)
        \pause
        \item isomorphism testing
        \pause
        \item classification data base of small surfaces
    \end{itemize}

    \pause
    Not yet implemented:
    \begin{itemize}
        \pause
        \item automorphism group
        \pause
        \item advanced properties (any wishes?)
    \end{itemize}
\end{frame}
