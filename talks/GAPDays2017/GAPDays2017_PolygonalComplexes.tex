\section{General simplicial surfaces}
\frame{\tableofcontents[currentsection]}

\begin{frame}{\uncover<3->{Triangular complexes}}
    We want to describe different structures:
    \uncover<2->{
        \begin{center}
            \begin{tikzpicture}[scale=2]
    \begin{scope}[scale=0.5,yshift=0.8cm]
                    % Coordinates of octahedron
                \def\len{2} % The length of the base
                \def\h{0.9}   % \h*\len is the hight of the apex above the 
                            % center of the square
                \coordinate (Mid) at (0,0);
                \coordinate (Right) at (0.5*\len,0.3*\len);
                \coordinate (Left) at (-0.75*\len,0.25*\len);
                \coordinate (Back) at ($(Right)+(Left)$);
                \coordinate (Center) at ($(Left)!0.5!(Right)$);
                \coordinate (Up) at ($(Center)+(0,\h*\len)$);
                \coordinate (Down) at ($(Center)+(0,-\h*\len)$);

                \filldraw[face] (Up) -- (Mid) -- (Left) -- cycle;
                \filldraw[face] (Up) -- (Mid) -- (Right) -- cycle;
                \filldraw[face] (Down) -- (Mid) -- (Left) -- cycle;
                \filldraw[face] (Down) -- (Mid) -- (Right) -- cycle;
                \draw[dashed] (Up) -- (Back) -- (Down);
                \draw[dashed] (Left) -- (Back) -- (Right);

 

    \end{scope}
    
    \begin{scope}[xshift=1cm]
        			    \coordinate (A) at (0,0);
			    \coordinate (B) at (1.7,0.5);
			    \coordinate (C) at (1.3,1.4);
			    \coordinate (D) at (0.5,1.5);
			    \coordinate (E) at (1,0.7);
			
			    % Take care to draw the faces in the back first
                            \draw[face,edge]
                                (A) -- (B) -- (C) -- cycle
                                (A) -- (C) -- (D) -- cycle;
                            \draw[face,edge]
                                (A) -- (B) -- (E) -- cycle
                                (A) -- (E) -- (D) -- cycle;
                            \draw[edge, dashed] (A) -- (C);

    \end{scope}

    \begin{scope}[xshift=3.5cm]
        			    \coordinate (A) at (0,0);
			    \coordinate (B) at (1.3,0.4);
			    \coordinate (C) at (0.4,1.3);
			
			    \filldraw[face] (A) -- (B) to[bend right=45] (C) -- cycle;
			    \filldraw[face] (A) -- (B) to[bend left=45] (C) -- cycle;


    \end{scope}
\end{tikzpicture}


        \end{center}
    }
    \uncover<3->{
        $\leadsto$ \textbf{triangular complexes}
    }
    \begin{itemize}
        \item<4-> sets of vertices, edges and faces
        \item<5-> incidence relation between them
        \item<6-> every face is a triangle
        \item<7-> every vertex lies in an edge \uncover<8->{and every edge lies in a face}
    \end{itemize}
\end{frame}


\begin{frame}{Isomorphism testing}
    Incidence structure can be interpreted as a coloured graph:
    \pause
    \begin{center}
        \begin{tikzpicture}[vertexNode/.style={circle,
            fill=\colorVertexNode}, node distance=0.5]
            % Edge nodes
            \node[edgeBackground] (E6) {6};
            \node[edgeBackground] (E8) [right=of E6] {8};
            \node[edgeBackground] (E9) [right=of E8] {9};
            \node[edgeBackground] (E10) [right=of E9] {10};
            \node[edgeBackground] (E12) [right=of E10] {12};
            \node[edgeBackground] (E13) [right=of E12] {13};

            % Face nodes
            \node[fill=\colorFace] (F1) [above=of E8] {1}
                edge (E6) edge (E8) edge (E9);
            \node (Help) at ($(E10)!0.5!(E12)$) {};
            \node[fill=\colorFace] (F4) [above=of Help] {4}
                edge (E9) edge (E10) edge (E12) edge (E13);

            % Vertex nodes
            \node[vertexNode] (V7) [below=of Help] {7}
                edge (E10) edge (E13);
            \node[vertexNode] (V11) [right=of V7] {11}
                edge (E13) edge (E12);
            \node[vertexNode] (V5) [left=of V7] {5}
                edge (E8) edge (E9) edge (E12);
            \node[vertexNode] (V3) [left=of V5] {3}
                edge (E6) edge (E9) edge (E10);
            \node[vertexNode] (V2) [left=of V3] {2}
                edge (E6) edge (E8);
        \end{tikzpicture}
    \end{center}
    \pause
    $\leadsto$ reduce to graph isomorphism problem

    \pause
    $\leadsto$ can be solved quite easily by \texttt{Nauty} (McKay, Piperno)

    \pause
    Interfaced by NautyTracesInterface (by Gutsche, Niemeyer, Schweitzer)
    \begin{itemize}
        \pause
        \item direct C--interface without writing files
        \pause
        \item also returns automorphism group
    \end{itemize}
\end{frame}


\begin{frame}{General properties}
    \pause
    Some properties can be computed for all polygonal complexes:
    \pause
    \begin{itemize}
        \item Connectivity
        \pause
        \item Euler--Characteristic
    \end{itemize}
    \pause
    \textit{Orientability} is \textbf{not} one of them.
    \pause
    Counterexample:
        \begin{center}
            \begin{tikzpicture}
                \coordinate (A) at (0,0);
                \coordinate (B) at (0,1.5);
                \coordinate (C) at (-0.7,0.4);
                \coordinate (D) at (0.8,0.4);
                \coordinate (E) at (0.9,0.8);
                			
                % draw back face first
                \filldraw[face] (A) -- (B) -- (E) -- cycle;
                % Now draw front faces
                \filldraw[face] (A) -- (B) -- (C) -- cycle;
                \filldraw[face] (A) -- (B) -- (D) -- cycle;
                % Draw dashed line
                \draw[dashed] (A) -- (E);
	    \end{tikzpicture}
        \end{center}
    \pause
    $\Rightarrow$ every edge lies in at most two faces (for well--definedness)
    
    \pause
    $\leadsto$ \textbf{ramified polygonal surfaces}
\end{frame}


\begin{frame}{Why ramified?}
    \pause
    Typical example of ramified polygonal surface:
    \pause
    \begin{center}
        \begin{tikzpicture}
                % First tetrahedron is ABCD, second one is DEFG
                \coordinate (A) at (-3,-1);
                \coordinate (B) at (-1,-1.5);
                \coordinate (C) at (-2,1);
                \coordinate (D) at (0,0);
                \coordinate (E) at (3,-0.1);
                \coordinate (F) at (1.5,2);
                \coordinate (G) at (1.2,1);
                
                \filldraw[face] (A) -- (B) -- (C) -- cycle;
                \filldraw[face] (B) -- (C) -- (D) -- cycle;
                \filldraw[face] (D) -- (F) -- (E) -- cycle;
                
                \draw[dashed] (A) -- (D);
                \draw[dashed] (E) -- (G);
                \draw[dashed] (D) -- (G);
                \draw[dashed] (F) -- (G);
        \end{tikzpicture}
    \end{center}
    \pause
    $\Rightarrow$ It is not a surface -- there is a \textit{ramification} at 
        the central vertex
    
    \pause
    A \textbf{polygonal surface} does not have these ramifications.
\end{frame}


