\section{Abstract folding}
\frame{\tableofcontents[currentsection]}

\begin{frame}{What kind of folding?}
    \pause
    There are many different kinds of folding (e.\,g. Origami)

    \pause
    Here:
    \begin{itemize}
        \pause
        \item Folding of surface in $\R^3$
        \pause
        \item Fold only at given edges (no introduction of new folding edges)
        \pause
        \item Folding should be rigid (no curvature)
    \end{itemize}

    \pause
    Goal: Classify possible folding patterns (given a net)

    \pause
    \begin{center}
        \only<1-7>{
            \phantom{ 
                \movie[ height = 0.4\textwidth, width = 0.4\textwidth, autostart, loop ]{}{TorusNetz.mp4}
            }
        }
        \only<8>{
            \movie[ height = 0.4\textwidth, width = 0.4\textwidth, autostart, loop ]{}{TorusNetz.mp4}
        }
    \end{center}

\end{frame}


\begin{frame}{Why are embeddings hard?}
    \pause
    Ideally, we would like to have embeddings.
    
    \pause
    But we want to define folding independently from an embedding, since:

    \begin{itemize}
        \pause
        \item They are very hard to compute (even for small examples)
        \pause
        \item We can only show foldability for specific small examples
            \begin{itemize}
                \pause
                \item Usually using regularity (like crystallographic symmetry)
                \pause
                \item No general method
            \end{itemize}
        \pause
        \item It is very hard to define iterated folding in an embedding
    \end{itemize}

    \pause
    \begin{center}
        \only<1-8>{
            \phantom{
                \movie[ height = 0.4\textwidth, width = 0.4\textwidth, autostart, loop ]{}{TorusNetz.mp4}
            }
        }
        \only<9>{
            \movie[ height = 0.4\textwidth, width = 0.4\textwidth, autostart, loop ]{}{TorusNetz.mp4}
        }
    \end{center}

\end{frame}


\begin{frame}{Folding without embedding}
    \pause
    Central idea:
    \begin{itemize}
        \pause
        \item Don't model folding process (needs embedding)
        \pause
        \item Describe starting and final folding state
            \begin{itemize}
                \pause
                \item Only consider changes in the topology
                    \pause (like identification of faces)
                \pause
                \item allows abstraction from embedding
            \end{itemize}
    \end{itemize}

    % Use pentagon-flippy to illustrate incidence rigidity

    \pause
    $\leadsto$ Incidence geometry (polygonal complex/surface)

    \begin{itemize}
        \pause
        \item Captures some folding restrictions \pause (rigidity of tetrahedron)
        \pause
        \item Still needs a lot of refinement
    \end{itemize}
\end{frame}

\newcommand{\colFaceA}{\colorFaceA}
\newcommand{\colFaceB}{\colorFaceB}
\newcommand{\colFaceC}{\colorFaceC}

\begin{frame}{Important properties of folding}
    \begin{itemize}
        \item<2-> The class of surfaces is not closed under folding
        \item<3-> Folding can be undone by \textit{unfolding}
        \item<4-> Identification of two faces might force identification of two other faces
            \begin{itemize}
                \item<7-> Can apply to arbitrary many faces 
                \item<10-> The forced identification is not unique
                \item<14-|handout:2>[$\Rightarrow$] Identify only two faces at a time
                    \begin{itemize}
                        \item<16-|handout:2>[$\leadsto$] Relax the rigidity--constraint:
                        \item<17-|handout:2> Allow non--rigid configurations as transitional states
                    \end{itemize}
            \end{itemize}
    \end{itemize}

    \begin{overlayarea}{\textwidth}{0.3\textwidth}
        \begin{center}
            \only<5-7,14|handout:0>{
                % First identification
                \begin{tikzpicture}
                    \def\Hdist{1.5}
	    	    \def\Vdist{1}
			
		    \coordinate (A) at (-\Hdist,0);
		    \coordinate (B) at (0,\Vdist);
		    \coordinate (C) at (\Hdist,0);
		    \coordinate (D) at (0,-\Vdist);
                    \draw (A) -- (B) -- (C) -- (D) -- cycle;

		    \foreach \p in {A,B,C,D}
		        \fill [\vertexColor] (\p) circle (1.5pt);
                    \planEdge{\colorRed,thick}{($(B)!0.5!(A)$)}{(barycentric cs:A=2,B=1,C=1,D=1)}{($(D)!0.5!(A)$)}
                    \uncover<6-7>{
                        \planEdge{\colorRed,thick}{($(B)!0.5!(C)$)}{(barycentric cs:A=1,B=1,C=2,D=1)}{($(D)!0.5!(C)$)}
                    }
                \end{tikzpicture}
            }

            % Arbitrary many faces
            \only<8-10|handout:0>{
                \begin{tikzpicture}
		    \foreach \i/\j in {1/0, 2/1, 3/2, 4/3, 5/4}
		    {
		        \filldraw [fill=black!9!white] (\j,0) -- (\i,0) -- (\i,1) -- (\j,1) -- cycle;
		        \filldraw [fill=black!9!white] (\j,0) -- (\i,0) -- (\i,-1) -- (\j,-1) -- cycle;
		    }
                    \uncover<9-10>{
                        \planEdge{\colorRed,thick}{(0.5,0.5)}{(0.75,0)}{(0.5,-0.5)}
	            }
                \end{tikzpicture}
            }

            % Picture of non-uniqueness:
            \only<11-13|handout:1>{
                \begin{tikzpicture}
                    \def\len{2}
                    \def\fak{0.7}
                    \coordinate (A) at (0,0);
                    \coordinate (B) at (\len,0);
                    \coordinate (C) at (\len,\len);
                    \coordinate (D) at (0,\len);
                    \coordinate (Back) at (0.2*\len,0.2*\len);

                    \behindPlane{Back}{A}{D}

                    \behindPlane<11>[\colFaceB]{Back}{A}{60:\fak*\len}
                    \behindPlane<12|handout:0>[\colFaceB]{Back}{A}{5:\fak*\len}
                    \behindPlane<13|handout:0>[\colFaceB]{Back}{A}{20:\fak*\len}

                    \behindPlane{Back}{A}{B}

                    \behindPlane<11>[\colFaceA]{Back}{B}{$(B)+(120:\fak*\len)$}
                    \behindPlane<12|handout:0>[\colFaceA]{Back}{B}{$(B)+(160:\fak*\len)$}
                    \behindPlane<13|handout:0>[\colFaceA]{Back}{B}{$(B)+(175:\fak*\len)$}

                    \planEdge[(Back)]{thick}{($(A)!0.5!(D)$)}{($(B)!0.7!(D)$)}{($(D)!0.5!(C)$)}

                    \behindPlane{Back}{C}{D}
                    \behindPlane{Back}{B}{C}

                \end{tikzpicture}
            }
            % 11: general picture
            % 12: First possible fold
            % 13: Second possible fold

            % Anomaly-picture
            \only<15-|handout:2>{
                \begin{tikzpicture}
		    \def\r{2}	
		    \def\eps{0.05}
				
		    \coordinate (A) at (-\r,0);
		    \coordinate (B) at (0,\eps);
		    \coordinate (C) at (\r,0);
		    \coordinate (D) at (0,-\eps);
			
		    \draw (D) -- (A) -- (B);
		    \draw (B) to [bend left=70] (C);
		    \draw (D) to [bend right=70] (C);
			
		    \foreach \point in {A,B,C,D}
		        \fill [\vertexColor] (\point) circle (1.5pt);
				
		    \draw [dashed] (0,0) circle (0.2);
                \end{tikzpicture}
            }
        \end{center}
    \end{overlayarea}
\end{frame}


\begin{frame}{How to define abstract folding?}
    \uncover<2->{We need to define two structures:}
    \begin{enumerate}
        \item<3-> A folding state
            \begin{itemize}
                \item<5-> Based on polygonal complexes
                \item<6-> Describe ``is folded together" by an equivalence relation
                \item<7-> Describe order of faces in folding state
            \end{itemize}
        \item<4-> The folding steps
            \begin{itemize}
                \item<8-> Only two faces at a time
                \item<9-> Explain ``unordered folding" (e.\,g. covering)
                \item<10-> Modify to include face order relations
            \end{itemize}
    \end{enumerate}
\end{frame}


\begin{frame}{Unordered Folding (Covering)}
    \begin{center}
        \begin{tikzpicture}
            \def\xOff{1.8}
            \def\yOff{1.5}

            %     B
            %   / | \
            % A   |   C
            %   \ | /
            %     D
            \coordinate (A) at (-\xOff,0);
            \coordinate (B) at (0,\yOff);
            \coordinate (C) at (\xOff,0);
            \coordinate (D) at (0,-\yOff);

            \uncover<3->{
                \filldraw[face] (A) -- (B) -- (D) -- cycle;
                \node at (barycentric cs:A=1,B=1,D=1) {$I$};
                \filldraw[face] (B) -- (C) -- (D) -- cycle;
                \node at (barycentric cs:B=1,C=1,D=1) {$II$};

                \foreach \p in {A,B,C,D}
                    \fill[\vertexColor] (\p) circle (1.5pt);
            }

            \uncover<5-7|handout:1>{
                \draw[\colorRed,thick,->] (barycentric cs:A=1,B=2,D=2) to[bend left] (barycentric cs:B=2,C=1,D=2);
                \draw[\colorRed,thick,->] ($(A)!0.5!(B)$) to[bend right] ($(B)!0.5!(C)$);
                \draw[\colorRed,thick,->] ($(A)!0.5!(D)$) to[bend left] ($(D)!0.5!(C)$);
            }
        \end{tikzpicture}
    \end{center}

    \begin{overprint}
        \onslide<1-7|handout:1>
        
        \uncover<2-7>{
            Why do we need more than a polygonal complex?
        }

        \uncover<4-7>{
            Naive folding definition: surjective map that respects incidence
        }

        \uncover<6-7>{
            Problem: Can't be unfolded
        }

        \uncover<7>{
            $\Rightarrow$ Folding state should not forget original structure
        }


        \onslide<8-16|handout:2>

        \uncover<8-16>{
            Represent folding by equivalence relation
            \begin{itemize}
                \item<9-> Separate relation on vertices, edges and faces
                \item<10-> Two elements are equivalent if they are folded together
                \item<11-> If two edges are equivalent, 
                    \uncover<12->{then their vertices have to be as well}
                    \uncover<13->{(likewise for faces)}
                \item<14-> The vertices of an edge are not equivalent
                    \uncover<15->{(likewise for faces)}
                \item<16->[$\Rightarrow$] Unordered folding is coarsening of equivalence relation
            \end{itemize}
        }
    \end{overprint}
\end{frame}


\begin{frame}{How does folding work?}
    \begin{enumerate}
        \item<2-> Choose two faces that are not folded together
        \item<4-> Choose how to identify them \uncover<6->{(like $I \sim II$ and $1 \sim 4$)}
        \item<7-> Add those pairs to the equivalence relation
    \end{enumerate}

    \begin{center}
        \begin{tikzpicture}
            \def\xOff{1.8}
            \def\yOff{1.5}

            %     B
            %   / | \
            % A   |   C
            %   \ | /
            %     D
            \uncover<3->{
                \coordinate[label={[vertex]left:1}] (A) at (-\xOff,0);
                \coordinate[label={[vertex]above:2}] (B) at (0,\yOff);
                \coordinate[label={[vertex]right:4}] (C) at (\xOff,0);
                \coordinate[label={[vertex]below:3}] (D) at (0,-\yOff);

                \filldraw[face] (A) -- (B) -- (D) -- cycle;
                \node at (barycentric cs:A=1,B=1,D=1) {$I$};
                \filldraw[face] (B) -- (C) -- (D) -- cycle;
                \node at (barycentric cs:B=1,C=1,D=1) {$II$};

                \foreach \p in {A,B,C,D}
                    \fill[\vertexColor] (\p) circle (1.5pt);
            }

            \uncover<5->{
                \draw[\colorRed,thick,->] (barycentric cs:A=1,B=2,D=2) to[bend left] (barycentric cs:B=2,C=1,D=2);
                \draw[\colorRed,thick,->] ($(A)!0.5!(B)$) to[bend right] ($(B)!0.5!(C)$);
                \draw[\colorRed,thick,->] ($(A)!0.5!(D)$) to[bend left] ($(D)!0.5!(C)$);
            }
        \end{tikzpicture}
    \end{center}

    \uncover<8->{Only restriction:}

    \uncover<9->{Two vertices in an edge can't be identified}
    \uncover<10->{(slightly generalized)}
\end{frame}


\begin{frame}[fragile]{Limitation of unordered folding}
    \pause
    We can't work with ordering of faces:
    \pause
    \begin{center}
        \begin{tikzpicture}

            \newcommand{\plane}[3]{
                \def\len{1.5}
                \def\back{0.75}
                \def\xOff{0.5}
                \def\nOff{-0.3}
                
                \coordinate (Down) at (#1*\xOff,0);
                \coordinate (Up) at (#1*\xOff,\len);
                \coordinate (Back) at (\back*\xOff,\back*\xOff);

                \fill[#2] (Down) -- ($(Down)+(Back)$) -- ($(Up)+(Back)$) -- (Up) -- cycle;
                \node[#2] at (#1*\xOff,\nOff) {#3};
            }

            \plane{0}{\colFaceB}{1}
            \plane{1}{\colFaceA}{2}
            \plane{2}{\colFaceC}{3}

            % Second lines
            \begin{scope}[xshift=4cm]
                \plane{0}{\colFaceB}{1}
                \plane{1}{\colFaceC}{3}
                \plane{2}{\colFaceA}{2}
            \end{scope}
        \end{tikzpicture}
    \end{center}

    \pause
    Adding a linear order on each face equivalence class
    \pause is not enough:
    \pause
    \begin{center}
        \begin{tikzpicture}

            \newcommand{\plane}[3]{
                \def\len{1}
                \def\nScale{1.3}

                \coordinate (Back) at (0.25*\len,0.5*\len);
                \fill[#2] (0,0) -- (#1:\len) -- +(Back) -- (Back) -- cycle; 
                \node[#2] at (#1:\nScale*\len) {#3};
                \draw (0,0) -- (Back);
            }

            \plane{210}{\colFaceA}{1}
            \plane{-30}{\colFaceB}{2}
            \plane{90}{\colFaceC}{3}

            \fill (0,0) circle (1.5pt);

            \begin{scope}[xshift=4cm]
                \plane{210}{\colFaceA}{1}
                \plane{-30}{\colFaceC}{3}
                \plane{90}{\colFaceB}{2}

                \fill (0,0) circle (1.5pt);
            \end{scope}
        \end{tikzpicture}
    \end{center}

    \pause
    $\leadsto$ define order of faces around edges
    \pause
    (we will skip the details)
\end{frame}


\begin{frame}{Folding complex}
    \pause
    \begin{defi}
        A \textbf{folding complex}
        \pause
        is a polygonal complex together with
        \begin{enumerate}
            \pause
            \item An equivalence relation on vertices, edges and faces
                \pause (``is folded together")
            \pause
            \item A linear ordering on each face equivalence class
            \pause
            \item A cyclical ordering of the faces around each edge equivalence class
        \end{enumerate}
        \pause
        such that the orderings are compatible
        \pause
        (in an appropriate sense).
    \end{defi}

    \pause
    To identify faces with each other, we have to combine those orderings.
    \begin{itemize}
        \pause
        \item linear orderings get concatenated
        \pause
        \item cyclical orderings are opened at one point
            \pause
            and combined
        \pause
        \item[!!] compatibility is not easily transfered,
            \pause
            but can be calculated
    \end{itemize}

\end{frame}


\begin{frame}[fragile]{Changed definition of folding}
    \begin{overprint}
        \only<2-4|handout:0>{Before (no ordering):}
        \only<5->{Folding with ordering:}
    \end{overprint}

    \begin{enumerate}
        \item<3-> Choose two faces that are not folded together
        \item<4-> Choose how to identify them and extend the equivalence relation
        \item<9-> Choose the sides of the faces that will meet
            \uncover<10->{and modify the orderings}
    \end{enumerate}

    \uncover<6->{
        $\leadsto$ Each face has two sides
    }

    \begin{center}
        \begin{tikzpicture}
            \def\len{1.5}
            \def\wA{20}
            \def\wB{160}
            \def\nOff{10}

            % angle, colour
            \newcommand{\plane}[2]{
                \coordinate (Out) at (#1:\len);
                \coordinate (Back) at (0.2*\len,0.2*\len);

                \fill[#2] (0,0) -- (Out) -- +(Back) -- (Back) -- cycle;
                \draw (0,0) -- (Back);
            }

            \uncover<7->{
                \plane{\wA}{\colFaceA}
                \node[\colFaceA] at (\wA+2*\nOff:\len) {$1^+$};
                \node[\colFaceA] at (\wA-\nOff:\len) {$1^-$};

                \plane{\wB}{\colFaceB}
                \node[\colFaceB] at (\wB+\nOff:0.9*\len) {$2^+$};
                \node[\colFaceB] at (\wB-4*\nOff:0.8*\len) {$2^-$};

                \fill (0,0) circle (1.5pt);
            }

            \uncover<8->{
                \planEdge{}{(\wB-\nOff: 0.5*\len)}{(90:0.4*\len)}{(\wA+\nOff:0.5*\len)}
            }

            \begin{scope}[xshift=4cm]
                \uncover<8->{
                    \def\wA{80}
                    \def\wB{100}

                    \plane{\wB}{\colFaceB}
                    \node[\colFaceB] at (\wB+\nOff:0.9*\len) {$2^+$};
                    
                    \plane{\wA}{\colFaceA}
                    \node[\colFaceA] at (\wA-2*\nOff:\len) {$1^-$};

                    \fill (0,0) circle (1.5pt);
                }
            \end{scope}

        \end{tikzpicture}
    \end{center}

    \uncover<11->{
        $\Rightarrow$ Define folding by two face sides (\textbf{folding plan})
    }

    \uncover<12->{
        $\leadsto$ Allows reversible (un)folding
    }
\end{frame}


\begin{frame}[fragile]{Structure of multiple foldings}
    \uncover<2->{
        With folding plans we can perform the same folding in different folding complexes
    }
    \begin{center}
        \begin{tikzpicture}
            \def\len{1.5}
            \def\off{10}

            %  B
            %   \
            %    Z -- C -- D
            %   /
            %  A
            \begin{scope}
                \coordinate (Z) at (0,0);
                \coordinate (A) at (-120:\len);
                \coordinate (B) at (120:\len);
                \coordinate (C) at (\len,0);
                \coordinate (D) at (2*\len,0);
                \coordinate (Back) at (-0.5*\len,0.25*\len);


                \uncover<3->{
                    \behindPlane{Back}{Z}{C}
                    \behindPlane{Back}{Z}{A}
                    \behindPlane{Back}{Z}{B}
                    \behindPlane{Back}{C}{D}
                    \foreach \p in {Z,A,B,C,D}
                        \fill[\vertexColor] (\p) circle (\vSize);
                }
                \uncover<5->{
                    \planEdge[(Back)]{\colorRed,thick}{(120+\off:0.5*\len)}{(180:0.5*\len)}{(-120-\off:0.5*\len)}
                }
            \end{scope}

            \begin{scope}[xshift=6cm]
                \def\pOff{0.1}

                \coordinate (Z) at (0,0);
                \coordinate (A) at (-120:\len);
                \coordinate (B) at (120:\len);
                \coordinate (C) at (\len,0);
                \coordinate (CAlt) at ($(C)+(0,0.5*\pOff)$);
                \coordinate (D) at (0.2*\len,\pOff);
                \coordinate (Back) at (-0.5*\len,0.25*\len);

                \uncover<4->{
                    \behindPlane{Back}{Z}{C}
                    \behindPlane{Back}{C}{$(C)+(0,\pOff)$}
                    \behindPlane{Back}{$(C)+(0,\pOff)$}{D}
                    \behindPlane{Back}{Z}{A}
                    \behindPlane{Back}{Z}{B}
                    \foreach \p in {A,B,Z,CAlt,D}
                        \fill[\vertexColor] (\p) circle (\vSize);
                }
                \uncover<6->{
                    \planEdge[(Back)]{\colorRed,thick}{(120+\off:0.5*\len)}{(180:0.5*\len)}{(-120-\off:0.5*\len)}
                }
            \end{scope}
        \end{tikzpicture}
    \end{center}

    \uncover<7->{
        $\leadsto$ more structure on the set of possible foldings
    }
\end{frame}


\begin{frame}[fragile]{Folding graph}
    \begin{itemize}
        \item<2->Vertices are folding complexes (modelling folding states)
        \item<3->Edges are folding plans connecting two folding complexes
    \end{itemize}

    \tikzset{graphVertex/.style={circle,fill=gray!70!white,inner sep=1.5pt}} %TODO make a better circle
    \begin{center}
        \begin{tikzpicture}[graphEdge/.style={thick}]
            \def\len{1.2}
            \def\off{10}
            \def\dist{0.5}

            \coordinate (Z) at (0,0);
            \coordinate (Back) at (0.4*\len,0.2*\len);
            \foreach \i in {0,1,2}
                \coordinate (P\i) at (120*\i:\len);


            \def\fgA{\colorGraphB}
            \def\fgB{\colorGraphA}
            \def\fgC{\colorGraphC}

            \planEdge<5->[(Back)]{graphEdge,\fgB}{(120+\off:0.5*\len)}{(180:\dist*\len)}{(240-\off:0.5*\len)}
            \behindPlane<4->{Back}{Z}{P2}
            \behindPlane<4->{Back}{Z}{P1}
            \planEdge<5->[(Back)]{graphEdge,\fgC}{(-120+\off:0.5*\len)}{(-60:\dist*\len)}{(-\off:0.5*\len)}
            \behindPlane<4->{Back}{Z}{P0}
            \planEdge<5->[(Back)]{graphEdge,\fgA}{(\off:0.5*\len)}{(60:\dist*\len)}{(120-\off:0.5*\len)}

            \uncover<4->{
                \foreach \i in {2,1,0}{
                    \fill[\vertexColor] (P\i) circle (1.5pt);
                }
            }

            \begin{scope}[xshift=3cm]
                \coordinate (Z) at (0,0);
                \coordinate (Back) at (0.4*\len,0.2*\len);
                \coordinate (P0) at (60-0.5*\off:\len);
                \coordinate (P1) at (60+0.5*\off:\len);
                \coordinate (P2) at (240:\len);

                \uncover<6->{
                    \planEdge[(Back)]{graphEdge,\fgB}{(240-\off:0.5*\len)}{(150:0.4*\len)}{(60+2*\off:0.5*\len)}
                    \behindPlane{Back}{Z}{P1}
                    \behindPlane{Back}{Z}{P2}
                    \behindPlane{Back}{Z}{P0}
                    \planEdge[(Back)]{graphEdge,\fgC}{(240+\off:0.5*\len)}{(-30:0.4*\len)}{(60-2*\off:0.5*\len)}

                    \foreach \i in {0,1,2}
                        \fill[\vertexColor] (P\i) circle (1.5pt);
                }
            \end{scope}

            % Second part
            \begin{scope}[xshift=7cm]
                \coordinate[graphVertex] (Top) at (0,\len);
                \node[graphVertex] (Mid) [below=of Top] {};
                \node[graphVertex] (MidLeft) [left=of Mid] {};
                \node[graphVertex] (MidRight) [right=of Mid] {};
                \node[graphVertex] (DownLeft) [below=of MidLeft] {};
                \node[graphVertex] (Down) [below=of Mid] {};
                \node[graphVertex] (DownRight) [below=of MidRight] {};

                \uncover<6->{
                    \draw[graphEdge, \fgA] (Top) -- (MidLeft);
                    \draw[graphEdge, \fgB] (Top) -- (Mid);
                    \draw[graphEdge, \fgC] (Top) -- (MidRight);
                }
                \uncover<7->{
                    \draw[graphEdge, \fgB] (MidLeft) -- (DownLeft);
                    \draw[graphEdge, \fgC] (MidLeft) -- (Down);
                }
                \uncover<8->{
                    \draw[graphEdge, \fgA] (Mid) -- (DownLeft);
                    \draw[graphEdge, \fgC] (Mid) -- (DownRight);
                    \draw[graphEdge, \fgA] (MidRight) -- (Down);
                    \draw[graphEdge, \fgB] (MidRight) -- (DownRight);
                }
            \end{scope}
        \end{tikzpicture}
    \end{center}

    % Second diagram
    \begin{center}
        \begin{tikzpicture}[graphEdge/.style={thick}]
            \def\fgA{\colorGraphAlpha}
            \def\fgB{\colorGraphBeta}
            \def\fgC{\colorGraphGamma}
            \def\fgD{\colorGraphDelta}

            % First picture
            \def\len{1.2}
            \def\off{10}
            \def\scal{0.4}

            \foreach \i in {0,1,2,3}
                \coordinate (P\i) at (\i*\len,0);

            \coordinate (Back) at (0.4*\len,0.2*\len);

            \planEdge<10->[(Back)]{graphEdge,\fgB}{($(P1)+(180+\off:\scal*\len)$)}{($(P1)+(270:\scal*\len)$)}{($(P1)+(-\off:\scal*\len)$)}
            \planEdge<10->[(Back)]{graphEdge,\fgD}{($(P2)+(180+\off:\scal*\len)$)}{($(P2)+(270:\scal*\len)$)}{($(P2)+(-\off:\scal*\len)$)}
            \behindPlane<9->{Back}{P0}{P1}
            \behindPlane<9->{Back}{P1}{P2}
            \behindPlane<9->{Back}{P2}{P3}
            \planEdge<10->[(Back)]{graphEdge,\fgA}{($(P1)+(180-\off:\scal*\len)$)}{($(P1)+(90:\scal*\len)$)}{($(P1)+(\off:\scal*\len)$)}
            \planEdge<10->[(Back)]{graphEdge,\fgC}{($(P2)+(180-\off:\scal*\len)$)}{($(P2)+(90:\scal*\len)$)}{($(P2)+(\off:\scal*\len)$)}


            \uncover<9->{
                \foreach \i in {0,1,2,3}
                    \fill[\vertexColor] (P\i) circle (1.5pt);
            }

            \begin{scope}[yshift=-1.3cm]
                \def\above{0.1}

                \coordinate (P1) at (\len,0);
                \coordinate (P1Alt) at ($(P1)+(0,0.5*\above)$);
                \coordinate (P2) at (2*\len,0);
                \coordinate (P3) at (3*\len,0);
                \coordinate (P0) at (1.8*\len,\above);

                \coordinate (Back) at (0.4*\len,0.2*\len);
                
                \planEdge<11->[(Back)]{graphEdge,\fgD}
                    {($(P2)+(180+\off:\scal*\len)$)}
                    {($(P2)+(270:\scal*\len)$)}
                    {($(P2)+(-\off:\scal*\len)$)}
                \behindPlane<11->{Back}{P1}{P2}
                \behindPlane<11->{Back}{P2}{P3}
                \behindPlane<11->{Back}{P1}{$(P1)+(0,\above)$}
                \behindPlane<11->{Back}{$(P1)+(0,\above)$}{P0}
                \planEdge<11->[(Back)]{graphEdge,\fgC,dashed}
                    {($(P2)+(180-\off:\scal*\len)$)}
                    {($(P2)+(90:\scal*\len)$)}
                    {($(P2)+(\off:\scal*\len)$)}
                
                \uncover<11->{
                    \foreach \p in {P1Alt,P0,P2,P3}
                        \fill[\vertexColor] (\p) circle (1.5pt);
                }
            \end{scope}


            \begin{scope}[xshift=6cm, yshift=-0.4cm]
                \coordinate[graphVertex] (B1) at (-\len,-\len);
                \foreach \i/\j in {1/2,2/3,3/4,4/5,5/6}{
                    \node[graphVertex] (B\j) [right=of B\i] {};
                    \coordinate (H\i\j) at ($(B\i)!0.5!(B\j)$);
                }
                \foreach \i/\j in {1/2,2/3,4/5,5/6}{
                    \node[graphVertex] (M\i\j) [above=of H\i\j] {};
                }

                \coordinate (M34) at ($(M23)!0.5!(M45)$);
                \node[graphVertex] (Top) [above=of M34] {};

                \uncover<11->{
                    \draw[graphEdge, \fgA] (Top) -- (M12);
                    \draw[graphEdge, \fgB] (Top) -- (M45);
                    \draw[graphEdge, \fgC] (Top) -- (M56);
                    \draw[graphEdge, \fgD] (Top) -- (M23);
                }

                \uncover<12->{
                    \draw[graphEdge, \fgD] (B2) -- (M12);
                    \draw[graphEdge, \fgA] (B2) -- (M23);
                    \draw[graphEdge, \fgB] (B5) -- (M56);
                    \draw[graphEdge, \fgC] (B5) -- (M45);
                }

                \uncover<13->{
                    \draw[graphEdge, \fgC,dashed] (B1) -- (M12);
                }

                \uncover<14->{
                    \draw[graphEdge, \fgB,dashed] (B3) -- (M23);
                    \draw[graphEdge, \fgD,dashed] (B4) -- (M45);
                    \draw[graphEdge, \fgA,dashed] (B6) -- (M56);
                }
            \end{scope}
        \end{tikzpicture}
    \end{center}
\end{frame}


\begin{frame}{Drawback of folding plans}
    \uncover<2->{
        Some foldings that ``should" be the same, aren't:
    }
    \begin{center}
        \begin{tikzpicture}
            \def\len{1.5}
            \def\off{10}

            %  B
            %   \
            %    Z -- C -- D
            %   /
            %  A
            \begin{scope}
                \coordinate (Z) at (0,0);
                \coordinate (A) at (-120:\len);
                \coordinate (B) at (120:\len);
                \coordinate (C) at (\len,0);
                \coordinate (D) at (2*\len,0);
                \coordinate (Back) at (0.5*\len,0.25*\len);

                \uncover<3->{
                    \behindPlane{Back}{Z}{A}
                    \behindPlane{Back}{Z}{B}
                    \behindPlane{Back}{Z}{C}
                    \behindPlane{Back}{C}{D}

                    \foreach \p in {Z,A,B,C,D}
                        \fill[\vertexColor] (\p) circle (\vSize);
                }
                \uncover<5->{
                    \planEdge[(Back)]{\colorRed,thick}{(120-\off:0.5*\len)}{(60:0.5*\len)}{(\off:0.5*\len)}
                }
            \end{scope}

            \begin{scope}[xshift=7cm]
                \def\pOff{0.1}

                \coordinate (Z) at (0,0);
                \coordinate (A) at (-120:\len);
                \coordinate (B) at (120:\len);
                \coordinate (C) at (\len,0);
                \coordinate (CAlt) at ($(C)+(0,0.5*\pOff)$);
                \coordinate (D) at (0.2*\len,\pOff);
                \coordinate (Back) at (0.5*\len,0.25*\len);

                \uncover<4->{
                    \behindPlane{Back}{Z}{A}
                    \behindPlane{Back}{Z}{B}
                    \behindPlane{Back}{Z}{C}
                    \behindPlane{Back}{C}{$(C)+(0,\pOff)$}
                    \behindPlane{Back}{$(C)+(0,\pOff)$}{D}
                    \foreach \p in {A,B,Z,CAlt,D}
                        \fill[\vertexColor] (\p) circle (\vSize);
                }
                \uncover<6->{
                    \planEdge[(Back)]{\colorRed,thick}{(120-\off:0.5*\len)}{(60:0.5*\len)}{(\off:0.5*\len)}
                }
            \end{scope}
        \end{tikzpicture}
    \end{center}


    \begin{itemize}
        \item<7->[$\Rightarrow$]If you know the folding structure of a small complex,
            \uncover<8->{you can't easily find the folding structure of an extended complex}
        \item<9->[$\leadsto$]Folding plans are not optimal to model folding.
    \end{itemize}
\end{frame}



