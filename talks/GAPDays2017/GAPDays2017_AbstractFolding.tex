\section{Abstract folding}
\frame{\tableofcontents[currentsection]}

\begin{frame}{What kind of folding?}
    \pause
    There are many different kinds of folding (e.\,g. Origami)

    \pause
    Here:
    \begin{itemize}
        \pause
        \item Folding of surface in $\R^3$
        \pause
        \item Fold only at given edges (no introduction of new folding edges)
        \pause
        \item Folding should be rigid (no curvature)
    \end{itemize}

    \pause
    Goal: Classify possible folding patterns (given a net)

    \pause
    \begin{center}
        \only<1-7>{
            \phantom{ 
                \movie[ height = 0.4\textwidth, width = 0.4\textwidth, autostart, loop ]{}{TorusNetz.mp4}
            }
        }
        \only<8>{
            \movie[ height = 0.4\textwidth, width = 0.4\textwidth, autostart, loop ]{}{TorusNetz.mp4}
        }
    \end{center}

\end{frame}


\begin{frame}{Embeddings are very hard}
    \begin{itemize}
        \item<2-> At every point in time the folding process has to be embedded
        \item<3-> We can only show foldability for specific small examples
            \begin{itemize}
                \item<4-> Usually using regularity (like crystallographic symmetry)
                \item<5-> No general method
            \end{itemize}
        \item<6-> It is very hard to define iterated folding in an embedding
    \end{itemize}

    \begin{center}
        \only<1-6>{
            \phantom{
                \movie[ height = 0.4\textwidth, width = 0.4\textwidth, autostart, loop ]{}{TorusNetz.mp4}
            }
        }
        \only<7>{
            \movie[ height = 0.4\textwidth, width = 0.4\textwidth, autostart, loop ]{}{TorusNetz.mp4}
        }
    \end{center}

\end{frame}


\begin{frame}{Folding without embedding}
    \pause
    Central idea:
    \begin{itemize}
        \pause
        \item Don't model folding process (needs embedding)
        \pause
        \item Describe starting and final folding state
            \begin{itemize}
                \pause
                \item Only consider changes in the topology
                    \pause (like identification of faces)
                \pause
                \item allows abstraction from embedding
            \end{itemize}
    \end{itemize}

    % Use pentagon-flippy to illustrate incidence rigidity

    \pause
    $\leadsto$ Incidence geometry (polygonal complex/surface)

    \begin{itemize}
        \pause
        \item Captures some folding restrictions \pause (rigidity of tetrahedron)
        \pause
        \item Still needs a lot of refinement
    \end{itemize}
\end{frame}

\newcommand{\colFaceA}{\colorFaceA}
\newcommand{\colFaceB}{\colorFaceB}
\newcommand{\colFaceC}{\colorFaceC}


\begin{frame}[fragile]{More than a triangular complex}
    \begin{itemize}
        \pause
        \item Concept should allow reversible folding
        \pause
        \item We need an ordering of the faces:
            \begin{center}
                \begin{tikzpicture}

                    \newcommand{\plane}[3]{
                        \def\len{1.5}
                        \def\back{0.75}
                        \def\xOff{0.5}
                        \def\nOff{-0.3}
                
                        \coordinate (Down) at (#1*\xOff,0);
                        \coordinate (Up) at (#1*\xOff,\len);
                        \coordinate (Back) at (\back*\xOff,\back*\xOff);

                        \fill[#2] (Down) -- ($(Down)+(Back)$) -- ($(Up)+(Back)$) -- (Up) -- cycle;
                        \node[#2] at (#1*\xOff,\nOff) {#3};
                    }

                    \plane{0}{\colFaceB}{1}
                    \plane{1}{\colFaceA}{2}
                    \plane{2}{\colFaceC}{3}

                    % Second lines
                    \begin{scope}[xshift=4cm]
                        \plane{0}{\colFaceB}{1}
                        \plane{1}{\colFaceC}{3}
                        \plane{2}{\colFaceA}{2}
                    \end{scope}
                \end{tikzpicture}
            \end{center}
        \pause
        \item Adding a linear order on each face equivalence class
            \pause is not enough:
            \pause
            \begin{center}
                \begin{tikzpicture}

                    \newcommand{\plane}[3]{
                        \def\len{1}
                        \def\nScale{1.3}

                        \coordinate (Back) at (0.25*\len,0.5*\len);
                        \fill[#2] (0,0) -- (#1:\len) -- +(Back) -- (Back) -- cycle; 
                        \node[#2] at (#1:\nScale*\len) {#3};
                        \draw (0,0) -- (Back);
                    }

                    \plane{210}{\colFaceA}{1}
                    \plane{-30}{\colFaceB}{2}
                    \plane{90}{\colFaceC}{3}

                    \fill (0,0) circle (1.5pt);

                    \begin{scope}[xshift=4cm]
                        \plane{210}{\colFaceA}{1}
                        \plane{-30}{\colFaceC}{3}
                        \plane{90}{\colFaceB}{2}

                        \fill (0,0) circle (1.5pt);
                    \end{scope}
                \end{tikzpicture}
            \end{center}
        \pause
        \item[$\leadsto$] \textbf{folding complex}
    \end{itemize}
\end{frame}


\begin{frame}[fragile]{How to describe folding?}
    \uncover<2->{
        Needs specification of two face sides:
    }

    \begin{center}
        \begin{tikzpicture}
            \def\len{1.5}
            \def\wA{20}
            \def\wB{160}
            \def\nOff{10}

                % angle, colour
                \newcommand{\plane}[2]{
                    \coordinate (Out) at (#1:\len);
                    \coordinate (Back) at (0.2*\len,0.2*\len);

                    \fill[#2] (0,0) -- (Out) -- +(Back) -- (Back) -- cycle;
                    \draw (0,0) -- (Back);
                }

            \uncover<3->{
                \plane{\wA}{\colFaceA}
                \node[\colFaceA] at (\wA+2*\nOff:\len) {$1^+$};
                \node[\colFaceA] at (\wA-\nOff:\len) {$1^-$};

                \plane{\wB}{\colFaceB}
                \node[\colFaceB] at (\wB+\nOff:0.9*\len) {$2^+$};
                \node[\colFaceB] at (\wB-4*\nOff:0.8*\len) {$2^-$};

                \fill (0,0) circle (1.5pt);
            }

            \uncover<4->{
                \planEdge{}{(\wB-\nOff: 0.5*\len)}{(90:0.4*\len)}{(\wA+\nOff:0.5*\len)}
            }

            \begin{scope}[xshift=4cm]
                \uncover<4->{
                    \def\wA{80}
                    \def\wB{100}

                    \plane{\wB}{\colFaceB}
                    \node[\colFaceB] at (\wB+\nOff:0.9*\len) {$2^+$};
                    
                    \plane{\wA}{\colFaceA}
                    \node[\colFaceA] at (\wA-2*\nOff:\len) {$1^-$};

                    \fill (0,0) circle (1.5pt);
                }
            \end{scope}

        \end{tikzpicture}
    \end{center}

    \uncover<5->{
        $\Rightarrow$ Describe folding by two face sides
    }

    \uncover<6->{
        $\leadsto$ \textbf{folding plan}
    }
\end{frame}


\begin{frame}{How does folding plan work?}
    \uncover<2->{
        Folding of two faces can force folding of other faces:
    }
    \begin{itemize}
        \item<5-> Can apply to arbitrary many faces
        \item<8-> The forced identification is not unique
        \item<12-|handout:2>[$\Rightarrow$] Identify only two faces at a time
            \begin{itemize}
                \item<14-|handout:2>[$\leadsto$] Relax the rigidity--constraint:
                \item<15-|handout:2> Allow non--rigid configurations as transitional states
            \end{itemize}
    \end{itemize}

    \begin{overlayarea}{\textwidth}{0.3\textwidth}
        \begin{center}
            \only<3-5,12->{ 
                % First identification
                \begin{tikzpicture}
                    \def\len{2}
                    \def\wink{40}
                    \def\off{2.5}
                    \def\bendAngle{70}

                    \coordinate (A) at (0,0);
                    \coordinate (B1) at (\wink:\len);
                    \coordinate (B2) at (\off:\len);
                    \coordinate (D1) at (-\wink:\len);
                    \coordinate (D2) at (-\off:\len);
                    \coordinate (C1) at ($(B1)+(D1)$);
                    \coordinate (C2) at ($(B2)+(D2)$);
                    \coordinate (Back) at (0.1*\len,0.2*\len);

                    
                    \uncover<3-5,12|handout:0>{
                        \behindPlane{Back}{A}{D1}
                        \behindPlane{Back}{D1}{C1}
                        \planEdge[(Back)]{\colorRed,thick}{($(B1)!0.5!(A)$)}
                            {(barycentric cs:A=2,B1=1,C1=1,D1=1)}{($(D1)!0.5!(A)$)}
                    }
                    \uncover<4-5|handout:0>{
                        \planEdge[(Back)]{\colorRed,thick}{($(B1)!0.5!(C1)$)}
                            {(barycentric cs:A=1,B1=1,C1=2,D1=1)}{($(D1)!0.5!(C1)$)}
                    }
                    \uncover<3-5,12|handout:0>{
                        \behindPlane{Back}{A}{B1}
                        \behindPlane{Back}{B1}{C1}

		        \foreach \p in {A,B1,C1,D1}
		            \fill [\vertexColor] (\p) circle (1.5pt);
                    }

                    % Second picture
                    \uncover<13-|handout:2>{
                        \behindPlane{Back}{A}{D2}
                        \behindPlane{Back}{A}{B2}
                        \curvedPlane{Back}{C2}{D2}{\bendAngle}
                        \curvedPlane{Back}{B2}{C2}{\bendAngle}
			
		        \foreach \point in {A,B2,C2,D2}
		            \fill [\vertexColor] (\point) circle (1.5pt);
                    }
                \end{tikzpicture}
            }

            % Arbitrary many faces
            \only<6-8|handout:0>{
                \begin{tikzpicture}
		    \foreach \i/\j in {1/0, 2/1, 3/2, 4/3, 5/4}
		    {
		        \filldraw [fill=black!9!white] (\j,0) -- (\i,0) -- (\i,1) -- (\j,1) -- cycle;
		        \filldraw [fill=black!9!white] (\j,0) -- (\i,0) -- (\i,-1) -- (\j,-1) -- cycle;
		    }
                    \uncover<7-8>{
                        \planEdge{\colorRed,thick}{(0.5,0.5)}{(0.75,0)}{(0.5,-0.5)}
	            }
                \end{tikzpicture}
            }

            % Picture of non-uniqueness:
            \only<9-11|handout:1>{
                \begin{tikzpicture}
                    \def\len{2}
                    \def\fak{0.7}
                    \coordinate (A) at (0,0);
                    \coordinate (B) at (\len,0);
                    \coordinate (C) at (\len,\len);
                    \coordinate (D) at (0,\len);
                    \coordinate (Back) at (0.2*\len,0.2*\len);

                    \behindPlane{Back}{A}{D}
                    \behindPlane{Back}{A}{B}

                    \behindPlane<9>[\colFaceB]{Back}{A}{60:\fak*\len}
                    \behindPlane<9>[\colFaceA]{Back}{B}{$(B)+(120:\fak*\len)$}

                    \behindPlane<10|handout:0>[\colFaceB]{Back}{A}{5:\fak*\len}
                    \behindPlane<10|handout:0>[\colFaceA]{Back}{B}{$(B)+(160:\fak*\len)$}

                    \behindPlane<11|handout:0>[\colFaceA]{Back}{B}{$(B)+(175:\fak*\len)$}
                    \behindPlane<11|handout:0>[\colFaceB]{Back}{A}{20:\fak*\len}

                    \planEdge[(Back)]{thick}{($(A)!0.5!(D)$)}{($(B)!0.7!(D)$)}{($(D)!0.5!(C)$)}

                    \behindPlane{Back}{C}{D}
                    \behindPlane{Back}{B}{C}

                \end{tikzpicture}
            }
            % 9: general picture
            % 10: First possible fold
            % 11: Second possible fold

        \end{center}
    \end{overlayarea}
\end{frame}


\begin{frame}[fragile]{Structure of multiple foldings}
    \uncover<2->{
        With folding plans we can perform the same folding in different folding complexes
    }
    \begin{center}
        \begin{tikzpicture}
            \def\len{1.5}
            \def\off{10}

            %  B
            %   \
            %    Z -- C -- D
            %   /
            %  A
            \begin{scope}
                \coordinate (Z) at (0,0);
                \coordinate (A) at (-120:\len);
                \coordinate (B) at (120:\len);
                \coordinate (C) at (\len,0);
                \coordinate (D) at (2*\len,0);
                \coordinate (Back) at (-0.5*\len,0.25*\len);


                \uncover<3->{
                    \behindPlane{Back}{Z}{C}
                    \behindPlane{Back}{Z}{A}
                    \behindPlane{Back}{Z}{B}
                    \behindPlane{Back}{C}{D}
                    \foreach \p in {Z,A,B,C,D}
                        \fill[\vertexColor] (\p) circle (\vSize);
                }
                \uncover<5->{
                    \planEdge[(Back)]{\colorRed,thick}{(120+\off:0.5*\len)}{(180:0.5*\len)}{(-120-\off:0.5*\len)}
                }
            \end{scope}

            \begin{scope}[xshift=6cm]
                \def\pOff{0.1}

                \coordinate (Z) at (0,0);
                \coordinate (A) at (-120:\len);
                \coordinate (B) at (120:\len);
                \coordinate (C) at (\len,0);
                \coordinate (CAlt) at ($(C)+(0,0.5*\pOff)$);
                \coordinate (D) at (0.2*\len,\pOff);
                \coordinate (Back) at (-0.5*\len,0.25*\len);

                \uncover<4->{
                    \behindPlane{Back}{Z}{C}
                    \behindPlane{Back}{C}{$(C)+(0,\pOff)$}
                    \behindPlane{Back}{$(C)+(0,\pOff)$}{D}
                    \behindPlane{Back}{Z}{A}
                    \behindPlane{Back}{Z}{B}
                    \foreach \p in {A,B,Z,CAlt,D}
                        \fill[\vertexColor] (\p) circle (\vSize);
                }
                \uncover<6->{
                    \planEdge[(Back)]{\colorRed,thick}{(120+\off:0.5*\len)}{(180:0.5*\len)}{(-120-\off:0.5*\len)}
                }
            \end{scope}
        \end{tikzpicture}
    \end{center}

    \uncover<7->{
        $\leadsto$ more structure on the set of possible foldings
    }
\end{frame}


\begin{frame}[fragile]{Folding graph}
    \begin{itemize}
        \item<2->Vertices are folding complexes (modelling folding states)
        \item<3->Edges are folding plans connecting two folding complexes
    \end{itemize}

    \begin{center}
        \begin{tikzpicture}[graphEdge/.style={thick}]
            \def\len{1.2}
            \def\off{10}
            \def\dist{0.5}

            \coordinate (Z) at (0,0);
            \coordinate (Back) at (0.4*\len,0.2*\len);
            \foreach \i in {0,1,2}
                \coordinate (P\i) at (120*\i:\len);


            \def\fgA{\colorGraphB}
            \def\fgB{\colorGraphA}
            \def\fgC{\colorGraphC}

            \planEdge<5->[(Back)]{graphEdge,\fgB}{(120+\off:0.5*\len)}{(180:\dist*\len)}{(240-\off:0.5*\len)}
            \behindPlane<4->{Back}{Z}{P2}
            \behindPlane<4->{Back}{Z}{P1}
            \planEdge<5->[(Back)]{graphEdge,\fgC}{(-120+\off:0.5*\len)}{(-60:\dist*\len)}{(-\off:0.5*\len)}
            \behindPlane<4->{Back}{Z}{P0}
            \planEdge<5->[(Back)]{graphEdge,\fgA}{(\off:0.5*\len)}{(60:\dist*\len)}{(120-\off:0.5*\len)}

            \uncover<4->{
                \foreach \i in {2,1,0}{
                    \fill[\vertexColor] (P\i) circle (1.5pt);
                }
            }

            \begin{scope}[xshift=3cm]
                \coordinate (Z) at (0,0);
                \coordinate (Back) at (0.4*\len,0.2*\len);
                \coordinate (P0) at (60-0.5*\off:\len);
                \coordinate (P1) at (60+0.5*\off:\len);
                \coordinate (P2) at (240:\len);

                \uncover<6->{
                    \planEdge[(Back)]{graphEdge,\fgB}{(240-\off:0.5*\len)}{(150:0.4*\len)}{(60+2*\off:0.5*\len)}
                    \behindPlane{Back}{Z}{P1}
                    \behindPlane{Back}{Z}{P2}
                    \behindPlane{Back}{Z}{P0}
                    \planEdge[(Back)]{graphEdge,\fgC}{(240+\off:0.5*\len)}{(-30:0.4*\len)}{(60-2*\off:0.5*\len)}

                    \foreach \i in {0,1,2}
                        \fill[\vertexColor] (P\i) circle (1.5pt);
                }
            \end{scope}

            % Second part
            \begin{scope}[xshift=7cm]
                \uncover<6->{
                    \coordinate[graphVertex] (Top) at (0,\len);
                    \node[graphVertex] (Mid) [below=of Top] {};
                    \node[graphVertex] (MidLeft) [left=of Mid] {};
                    \node[graphVertex] (MidRight) [right=of Mid] {};
                }
                \uncover<7->{
                    \node[graphVertex] (DownLeft) [below=of MidLeft] {};
                    \node[graphVertex] (Down) [below=of Mid] {};
                }
                \uncover<8->{
                    \node[graphVertex] (DownRight) [below=of MidRight] {};
                }

                \uncover<6->{
                    \draw[graphEdge, \fgA] (Top) -- (MidLeft);
                    \draw[graphEdge, \fgB] (Top) -- (Mid);
                    \draw[graphEdge, \fgC] (Top) -- (MidRight);
                }
                \uncover<7->{
                    \draw[graphEdge, \fgB] (MidLeft) -- (DownLeft);
                    \draw[graphEdge, \fgC] (MidLeft) -- (Down);
                }
                \uncover<8->{
                    \draw[graphEdge, \fgA] (Mid) -- (DownLeft);
                    \draw[graphEdge, \fgC] (Mid) -- (DownRight);
                    \draw[graphEdge, \fgA] (MidRight) -- (Down);
                    \draw[graphEdge, \fgB] (MidRight) -- (DownRight);
                }
            \end{scope}
        \end{tikzpicture}
    \end{center}

    % Second diagram
    \begin{center}
        \begin{tikzpicture}[graphEdge/.style={thick}]
            \def\fgA{\colorGraphAlpha}
            \def\fgB{\colorGraphBeta}
            \def\fgC{\colorGraphGamma}
            \def\fgD{\colorGraphDelta}

            % First picture
            \def\len{1.2}
            \def\off{10}
            \def\scal{0.4}

            \foreach \i in {0,1,2,3}
                \coordinate (P\i) at (\i*\len,0);

            \coordinate (Back) at (0.4*\len,0.2*\len);

            \planEdge<10->[(Back)]{graphEdge,\fgB}{($(P1)+(180+\off:\scal*\len)$)}{($(P1)+(270:\scal*\len)$)}{($(P1)+(-\off:\scal*\len)$)}
            \planEdge<10->[(Back)]{graphEdge,\fgD}{($(P2)+(180+\off:\scal*\len)$)}{($(P2)+(270:\scal*\len)$)}{($(P2)+(-\off:\scal*\len)$)}
            \behindPlane<9->{Back}{P0}{P1}
            \behindPlane<9->{Back}{P1}{P2}
            \behindPlane<9->{Back}{P2}{P3}
            \planEdge<10->[(Back)]{graphEdge,\fgA}{($(P1)+(180-\off:\scal*\len)$)}{($(P1)+(90:\scal*\len)$)}{($(P1)+(\off:\scal*\len)$)}
            \planEdge<10->[(Back)]{graphEdge,\fgC}{($(P2)+(180-\off:\scal*\len)$)}{($(P2)+(90:\scal*\len)$)}{($(P2)+(\off:\scal*\len)$)}


            \uncover<9->{
                \foreach \i in {0,1,2,3}
                    \fill[\vertexColor] (P\i) circle (1.5pt);
            }

            \begin{scope}[yshift=-1.3cm]
                \def\above{0.1}

                \coordinate (P1) at (\len,0);
                \coordinate (P1Alt) at ($(P1)+(0,0.5*\above)$);
                \coordinate (P2) at (2*\len,0);
                \coordinate (P3) at (3*\len,0);
                \coordinate (P0) at (1.8*\len,\above);

                \coordinate (Back) at (0.4*\len,0.2*\len);
                
                \planEdge<11->[(Back)]{graphEdge,\fgD}
                    {($(P2)+(180+\off:\scal*\len)$)}
                    {($(P2)+(270:\scal*\len)$)}
                    {($(P2)+(-\off:\scal*\len)$)}
                \behindPlane<11->{Back}{P1}{P2}
                \behindPlane<11->{Back}{P2}{P3}
                \behindPlane<11->{Back}{P1}{$(P1)+(0,\above)$}
                \behindPlane<11->{Back}{$(P1)+(0,\above)$}{P0}
                \planEdge<11->[(Back)]{graphEdge,\fgC,dashed}
                    {($(P2)+(180-\off:\scal*\len)$)}
                    {($(P2)+(90:\scal*\len)$)}
                    {($(P2)+(\off:\scal*\len)$)}
                
                \uncover<11->{
                    \foreach \p in {P1Alt,P0,P2,P3}
                        \fill[\vertexColor] (\p) circle (1.5pt);
                }
            \end{scope}


            \begin{scope}[xshift=6cm, yshift=-0.4cm]
                \uncover<13->{
                    \coordinate[graphVertex] (B1) at (-\len,-\len);
                }
                \foreach \i/\j/\pos in {1/2/12,2/3/14,3/4/14,4/5/12,5/6/14}{
                    \uncover<\pos->{
                        \node[graphVertex] (B\j) [right=of B\i] {};
                    }
                    \coordinate (H\i\j) at ($(B\i)!0.5!(B\j)$);
                }

                \uncover<11->{
                    \foreach \i/\j in {1/2,2/3,4/5,5/6}{
                        \node[graphVertex] (M\i\j) [above=of H\i\j] {};
                    }

                    \coordinate (M34) at ($(M23)!0.5!(M45)$);
                    \node[graphVertex] (Top) [above=of M34] {};
                }

                \uncover<11->{
                    \draw[graphEdge, \fgA] (Top) -- (M12);
                    \draw[graphEdge, \fgB] (Top) -- (M45);
                    \draw[graphEdge, \fgC] (Top) -- (M56);
                    \draw[graphEdge, \fgD] (Top) -- (M23);
                }

                \uncover<12->{
                    \draw[graphEdge, \fgD] (B2) -- (M12);
                    \draw[graphEdge, \fgA] (B2) -- (M23);
                    \draw[graphEdge, \fgB] (B5) -- (M56);
                    \draw[graphEdge, \fgC] (B5) -- (M45);
                }

                \uncover<13->{
                    \draw[graphEdge, \fgC,dashed] (B1) -- (M12);
                }

                \uncover<14->{
                    \draw[graphEdge, \fgB,dashed] (B3) -- (M23);
                    \draw[graphEdge, \fgD,dashed] (B4) -- (M45);
                    \draw[graphEdge, \fgA,dashed] (B6) -- (M56);
                }
            \end{scope}
        \end{tikzpicture}
    \end{center}
\end{frame}


\begin{frame}[fragile]{Larger graph}
    \uncover<2->{
        \begin{center}
                        \begin{tikzpicture}
                \def\len{1.2}
                \def\off{10}
                \def\scal{0.4}
                
                \coordinate (Z) at (0,0);
                \coordinate (A) at (-120:\len);
                \coordinate (B) at (120:\len);
                \coordinate (C) at (\len,0);
                \coordinate (D) at (2*\len,0);
                \coordinate (Back) at (0.5*\len,0.25*\len);

                \planEdge[(Back)]{\graphColorA,thick}{(120+\off:0.5*\len)}
                    {(180:0.5*\len)}{(-120-\off:0.5*\len)}
                \behindPlane{Back}{Z}{A}
                \behindPlane{Back}{Z}{B}
                \planEdge[(Back)]{\graphColorE,thick}{(-\off:0.5*\len)}
                    {(-60:0.5*\len)}{(-120+\off:0.5*\len)}
                \planEdge[(Back)]{\graphColorD,thick}{($(C)+(180+\off:\scal*\len)$)}
                    {($(C)+(-90:\scal*\len)$)}{($(C)+(-\off:\scal*\len)$)}
                \behindPlane{Back}{Z}{C}
                \behindPlane{Back}{C}{D}
                \planEdge[(Back)]{\graphColorB,thick}{(\off:0.5*\len)}
                    {(60:0.5*\len)}{(120-\off:0.5*\len)}
                \planEdge[(Back)]{\graphColorC,thick}{($(C)+(180-\off:\scal*\len)$)}
                    {($(C)+(90:\scal*\len)$)}{($(C)+(\off:\scal*\len)$)}

                \foreach \p in {Z,A,B,C,D}
                    \fill[\vertexColor] (\p) circle (\vSize);
            \end{tikzpicture}
 

        \end{center}
    }

    \uncover<3->{
        \begin{center}
            \begin{tikzpicture}[every edge/.style={draw,thick},node distance=1.5 and 0.9]
                \newcommand{\cA}{\graphColorA}
                \newcommand{\cB}{\graphColorB}
                \newcommand{\cC}{\graphColorC}
                \newcommand{\cD}{\graphColorD}
                \newcommand{\cE}{\graphColorE}
                
                \node[graphVertex] (M1) {};
                \foreach \i/\j in {1/2,2/3,3/4,4/5,5/6,6/7,7/8,8/9,9/10,10/11}{
                    \node[graphVertex] (M\j) [right=of M\i] {};
                }

                \coordinate (Help) at ($(M1)!0.5!(M2)$);
                \node[graphVertex] (N2) [below=of Help] {}
                    edge[\cA] (M1) edge[dashed,\cE] (M3);
                \node[graphVertex] (N1) [left=of N2] {}
                    edge[\cB] (M1) edge[dashed,\cE] (M2);
                \node[graphVertex] (N3) [right=of N2] {}
                    edge[\cB] (M3) edge[\cA] (M2) edge[\cD] (M4);
                \node[graphVertex] (N4) [right=of N3] {}
                    edge[dotted,\cC] (M4);
                \node[graphVertex] (N5) [right=of N4] {}
                    edge[dotted,\cA] (M5);
                \node[graphVertex] (N6) [right=of N5] {}
                    edge[\cE] (M5) edge[dashed,\cC] (M6);
                \node[graphVertex] (N7) [right=of N6] {}
                    edge[dashed,\cD] (M6) edge[\cB] (M7);
                \node[graphVertex] (N8) [right=of N7] {}
                    edge[dotted,\cA] (M7);
                \node[graphVertex] (N9) [right=of N8] {}
                    edge[dotted,\cD] (M8);
                \node[graphVertex] (N10) [right=of N9] {}
                    edge[\cC] (M8) edge[\cE] (M9) edge[\cA] (M10);
                \node[graphVertex] (N11) [right=of N10] {}
                    edge[dashed,\cB] (M9) edge[\cA] (M11);
                \node[graphVertex] (N12) [right=of N11] {}
                    edge[dashed,\cB] (M10) edge[\cE] (M11);

                    
                \node[graphVertex] (L1) [above=of M2] {}
                    edge[dashed,\cE] (M1) edge[\cA] (M3) edge[\cB] (M2);
                \phantom{\node[graphVertex] (M45) at ($(M4)!0.5!(M5)$) {}; };
                \node[graphVertex] (L2) [above=of M4] {}
                    edge[\cD] (M2) edge[dashed,\cC] (M5) edge[\cA] (M4) edge[\cE] (M6);
                \node[graphVertex] (L3) [above=of M6] {}
                    edge[\cD] (M3) edge[\cB] (M4) edge[\cE] (M8) edge[\cC] (M9);
                \phantom{\node[graphVertex] (M78) at ($(M7)!0.5!(M8)$) {}; };
                \node[graphVertex] (L4) [above=of M8] {}
                    edge[\cB] (M6) edge[\cA] (M8) edge[dashed,\cD] (M7) edge[\cC] (M10);
                \node[graphVertex] (L5) [above=of M10] {}
                    edge[\cA] (M9) edge[\cE] (M10) edge[dashed,\cB] (M11);


                \node[graphVertex] (K) [above=of L3] {}
                    edge[\cD] (L1) edge[\cB] (L2) edge[\cA] (L3) edge[\cE] (L4) edge[\cC] (L5);

            \end{tikzpicture}
        \end{center}
    }

\end{frame}


\begin{frame}{Drawback of folding plans}
    \uncover<2->{
        Some foldings that ``should" be the same, aren't:
    }
    \begin{center}
        \begin{tikzpicture}
            \def\len{1.5}
            \def\off{10}

            %  B
            %   \
            %    Z -- C -- D
            %   /
            %  A
            \begin{scope}
                \coordinate (Z) at (0,0);
                \coordinate (A) at (-120:\len);
                \coordinate (B) at (120:\len);
                \coordinate (C) at (\len,0);
                \coordinate (D) at (2*\len,0);
                \coordinate (Back) at (0.5*\len,0.25*\len);

                \uncover<3->{
                    \behindPlane{Back}{Z}{A}
                    \behindPlane{Back}{Z}{B}
                    \behindPlane{Back}{Z}{C}
                    \behindPlane{Back}{C}{D}

                    \foreach \p in {Z,A,B,C,D}
                        \fill[\vertexColor] (\p) circle (\vSize);
                }
                \uncover<5->{
                    \planEdge[(Back)]{\colorRed,thick}{(120-\off:0.5*\len)}{(60:0.5*\len)}{(\off:0.5*\len)}
                }
            \end{scope}

            \begin{scope}[xshift=7cm]
                \def\pOff{0.1}

                \coordinate (Z) at (0,0);
                \coordinate (A) at (-120:\len);
                \coordinate (B) at (120:\len);
                \coordinate (C) at (\len,0);
                \coordinate (CAlt) at ($(C)+(0,0.5*\pOff)$);
                \coordinate (D) at (0.2*\len,\pOff);
                \coordinate (Back) at (0.5*\len,0.25*\len);

                \uncover<4->{
                    \behindPlane{Back}{Z}{A}
                    \behindPlane{Back}{Z}{B}
                    \behindPlane{Back}{Z}{C}
                    \behindPlane{Back}{C}{$(C)+(0,\pOff)$}
                    \behindPlane{Back}{$(C)+(0,\pOff)$}{D}
                    \foreach \p in {A,B,Z,CAlt,D}
                        \fill[\vertexColor] (\p) circle (\vSize);
                }
                \uncover<6->{
                    \planEdge[(Back)]{\colorRed,thick}{(120-\off:0.5*\len)}{(60:0.5*\len)}{(\off:0.5*\len)}
                }
            \end{scope}
        \end{tikzpicture}
    \end{center}


    \begin{itemize}
        \item<7->[$\Rightarrow$]If you know the folding structure of a small complex,
            \uncover<8->{you can't easily find the folding structure of an extended complex}
        \item<9->[$\leadsto$]Folding plans are not optimal to model folding
    \end{itemize}
\end{frame}


\begin{frame}{Progress of abstract folding}
    \pause
    In development:
    \begin{itemize}
        \pause
        \item folding complex
        \pause
        \item folding plans
        \pause
        \item folding graph
    \end{itemize}

    \pause
    Missing:
    \begin{itemize}
        \pause
        \item better folding description
        \pause
        \item properties of folding graphs
    \end{itemize}
\end{frame}


